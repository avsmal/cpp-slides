\documentclass{beamer}

\usepackage{mooc}

\usetikzlibrary{arrows, decorations.markings, shapes}
\usetikzlibrary{positioning}

% The face style, can be changed

\title{{\bf Программирование на языке \langcpp\protect\\Лекция
3\protect\vspace{1em}\\}Константность}

\begin{document}
\begin{frame} 
  \titlepage
\end{frame}

\begin{frame}[fragile]{Определение констант}{}
    \begin{itemize}
        \item Ключевое слово \code{const} позволяет
            определять типизированные константы.
\begin{lstlisting}
double const pi = 3.1415926535;
int const day_seconds = 24 * 60 * 60;   
// массив констант
int const days[12] = {31, 28, 31,
                      30, 31, 30,
                      31, 31, 30,
                      31, 30, 31};
\end{lstlisting}
        \item Попытка изменить константные данные приводит к неопределённому
            поведению.
\begin{lstlisting}
    int * may = (int *) &days[4];
    *may = 30;
\end{lstlisting}
    \end{itemize}
\end{frame}

\begin{frame}[fragile]{Указатели и \code{const}}{}
В \langcpp можно определить как константный указатель,
так и указатель на константу:
\begin{lstlisting}
int a = 10;
const int * p1 = &a; // указатель на константу
int const * p2 = &a; // указатель на константу
*p1 = 20; // ошибка
p2  = 0;  // ОК

int * const p3 = &a; // константный указатель
*p3 = 30; // OK
p3  = 0;  // ошибка

// константный указатель на константу
int const * const p4 = &a;
*p4 = 30; // ошибка
p4  = 0;  // ошибка
\end{lstlisting}
\end{frame}

\begin{frame}[fragile]{Указатели и \code{const}}{}
    Можно использовать следующее правило:\\
    \begin{center}
    ``слово \code{const} делает неизменяемым тип слева от него''.
    \end{center}

\begin{lstlisting}
int a = 10;
int * p = &a;   

// указатель на указатель на const int
int const **  p1 = &p;

// указатель на константный указатель на int
int * const * p2 = &p;

// константный указатель на указатель на int
int ** const  p3 = &p;
\end{lstlisting}
\end{frame}

\begin{frame}[fragile]{Ссылки и \code{const}}{}
    \begin{itemize}
        \item Ссылка сама по себе является неизменяемой.
\begin{lstlisting}
int a = 10;
int & const b = a; // ошибка
int const & c = a; // ссылка на константу
\end{lstlisting}
        \item Использование константных ссылок позволяет избежать
            копирования объектов при передаче в функцию.
\begin{lstlisting}
Point midpoint(Segment const & s);
\end{lstlisting}
        \item По константной ссылке можно передавать rvalue.
\begin{lstlisting}
    Point p = midpoint(Segment(Point(0,0), 
                               Point(1,1)));
\end{lstlisting}
    \end{itemize}
\end{frame}

\begin{frame}[fragile]{Константные методы}{}
    \begin{itemize}
        \item Методы классов могут быть объявлены как \code{const}.
\begin{lstlisting}
struct IntArray {
    size_t size() const;
};
\end{lstlisting}
        \item Такие методы не могут менять поля объекта\\
            (тип \code{this}~--- указатель на \code{const}).

        \item У константных объектов (через указатель или ссылку на константу)
            можно вызывать только константные методы:

\begin{lstlisting}
IntArray const * p = foo();
p->resize(); // ошибка
\end{lstlisting}
        \item Внутри константных методов можно вызывать только константные методы.

    \end{itemize}
\end{frame}

\begin{frame}[fragile]{Две версии одного метода}{}
    \begin{itemize}
        \item Слово \code{const} является частью сигнатуры метода.
\begin{lstlisting}
size_t IntArray::size() const {return size_;}
\end{lstlisting}
        \item Можно определить две версии одного метода:
\begin{lstlisting}
struct IntArray {
    int get(size_t i) const { 
        return data_[i]; 
    }
    int & get(size_t i) { 
        return data_[i]; 
    }
private:
    size_t size_;
    int *  data_;
};
\end{lstlisting}

    \end{itemize}
\end{frame}

\begin{frame}[fragile]{Cинтаксическая и логическая константность}{}
\begin{itemize}
    \item Синтаксическая константность: константные методы
        не могут менять поля (обеспечивается компилятором).
    \item Логическая константность~--- нельзя менять те данные, 
        которые определяют состояние объекта.
\begin{lstlisting}
struct IntArray {
    void foo() const {
       // нарушение логической константности
        data_[10] = 1; 
    }
private:
    size_t size_;
    int *  data_;
};
\end{lstlisting}
\end{itemize}
    
\end{frame}


\begin{frame}[fragile]{Ключевое слово \code{mutable}}{}
Ключевое слово \code{mutable} позволяет определять поля,
которые можно изменять внутри константных методов:
\begin{lstlisting}
struct IntArray {
    size_t size() const { 
        ++counter_;
        return size_; 
    }

private:
    size_t size_;
    int *  data_;

    mutable size_t counter_;
};
\end{lstlisting}
\end{frame}

\end{document}
