\documentclass[aspectration=1610,t,12pt]{beamer}

\usepackage{mooc}

\title{{\bf Программирование на языке \langcpp\protect\\Лекция
7\protect\vspace{1em}\\}О продолжении курса}

\begin{document}
\begin{frame} 
  \titlepage
\end{frame}

\begin{frame}{Общая информация}
    \begin{itemize}
        \item Предполагается, что слушатели уже ознакомились с первой частью курса.
        \pitem Материал курса значительно более сложный, чем в первой части, некоторые концепции сложны для понимания и могут потребовать больше времени и усилий.
        \pitem Уточняющая информация к некоторым видео оформлена в виде текстовых степов. Мы ожидаем, что слушатели внимательно их прочтут и разберутся.
        \pitem Не ко всем темам можно подобрать разумные задачи, поэтому мы ожидаем от слушателей, что они будут самостоятельно экспериментировать и применять полученные знания.
    \end{itemize}
\end{frame}

\begin{frame}{О чём этот курс}{}
\begin{itemize}
	\item \myq{Продвинутые} возможности \langcpp, которые не попали в первую часть курса.
	\pitem Новые возможности из стандартов \langcpp11/14.
	\pitem Стандартная библиотека шаблонов.
	\pitem Обработка ошибок.
	\pitem Дополнительные главы:
	\begin{itemize}
	\item многопоточность, 
	\item библиотека boost, 
	\item метапрограммирование.
	\end{itemize} 
\end{itemize}
\end{frame}

\end{document}