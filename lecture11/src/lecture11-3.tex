\documentclass[aspectration=1610]{beamer}

\usepackage{mooc}

\title{{\bf Программирование на языке \langcpp\protect\\Лекция
11\protect\vspace{1em}\\}Метапрограммирование: основы}

\begin{document}
\begin{frame} 
  \titlepage
\end{frame}

\begin{frame}[fragile]{Метапрограммирование}
\begin{itemize}
    \item Метапрограммированием называют создание программ, \\которые порождают другие программы.
    
    \pitem Шаблоны \langcpp можно рассматривать как функциональный\\ язык
    для метапрограммирования.
    
    \pitem Метапрограммы \langcpp позволяют оперировать типами, \\шаблонами и значениями целочисленных типов.
    
    \pitem Метапрограммирование в \langcpp можно применять\\ для широкого круга задач:
    \begin{itemize}
        \item целочисленные compile-time вычисления,
        \item compile-time проверка ошибок,
        \item условная компиляция,
        \item генеративное программирование,
        \item \ldots
    \end{itemize}
    
    \pitem Для метапрограммирования существуют целые\\ библиотеки,
    например, MPL из boost.
\end{itemize}
\end{frame}

\begin{frame}[fragile]{Метафункции}
Метафункция~--- это шаблонный класс, 
который определяет \\имя типа \texttt{type} или целочисленную константу \texttt{value}.
\begin{itemize}
\item Аргументы метафункции~--- это аргументы шаблона.
\item Возвращаемое значение~--- это \texttt{type} или \texttt{value}.
\end{itemize}

Метафункции могут возвращать типы:
\begin{lstlisting}
template<typename T> 
struct AddPointer
{
    using type = T *; 
};
\end{lstlisting}
и значения целочисленных типов:
\begin{lstlisting}
template<int N> 
struct Square
{
    static int const value = N * N; 
};
\end{lstlisting}
\end{frame}


\begin{frame}[fragile]{Вычисления в compile-time}
\begin{lstlisting}
template<int N> 
struct Fact 
{
    static int const
        value = N * Fact<N - 1>::value;
};

template<> 
struct Fact<0> 
{
    static int const value = 1;
};

int main() 
{
    std::cout << Fact<10>::value << std::endl;
}
\end{lstlisting}

(Это вычисление можно реализовать через \code{constexpr}.) 
\end{frame}

%template<int N> struct Fib {
%    constexpr static int 
%        value = Fib<N - 1>::value + Fib<N - 2>::value;
%};
%template<> struct Fib<0> {
%    static const int value = 0;
%};
%template<> struct Fib<1> {
%    static const int value = 1;
%};
%int main() {
%    Fib <10>::value << std::endl;
%}

\begin{frame}[fragile]{Определение списка}
Шаблоны позволяют определять алгебраические типы данных.
\begin{lstlisting}
// определяем список
template <typename ... Types>
struct TypeList; 

// специализация по умолчанию
template <typename H, typename... T>
struct TypeList<H, T...> 
{
    using Head = H;
    using Tail = TypeList<T...>;
};

// специализация для пустого списка
template <>
struct TypeList<> { };
\end{lstlisting}
\end{frame}

\begin{frame}[fragile]{Длина списка}
\begin{lstlisting}
// вычисление длины списка
template<typename TL>
struct Length
{
    static int const value = 1 +
        Length<typename TL::Tail>::value;
};

template<>
struct Length<TypeList<>>
{
    static int const value = 0;
};

int main()
{
    using TL = TypeList<double, float, int, char>;
    std::cout << Length<TL>::value << std::endl;
    return 0;
}
\end{lstlisting}
\end{frame}

\begin{frame}[fragile]{Операции со списком}
\begin{lstlisting}
// добавление элемента в начало списка
template<typename H, typename TL>
struct Cons;

template<typename H, typename... Types>
struct Cons<H, TypeList<Types...>>
{
    using type = TypeList<H, Types...>;
};

// конкатенация списков
template<typename TL1, typename TL2>
struct Concat;

template<typename... Ts1, typename... Ts2>
struct Concat<TypeList<Ts1...>, TypeList<Ts2...>>
{
    using type = TypeList<Ts1..., Ts2...>;
};
\end{lstlisting}
\end{frame}

\begin{frame}[fragile]{Вывод списка}
\begin{lstlisting}
// вывод списка в поток os
template<typename TL>
void printTypeList(std::ostream & os)
{
    os << typeid(typename TL::Head).name() << '\n';
    printTypeList<typename TL::Tail>(os);
};

// вывод пустого списка
template<>
void printTypeList<TypeList<>>(std::ostream & os) {}

int main()
{
    using TL = TypeList<double, float, int, char>;
    printTypeList<TL>(std::cout);
    return 0;
}
\end{lstlisting}
\end{frame}

\end{document}
