\documentclass[aspectration=1610,t]{beamer}

\usepackage{mooc}

\title{{\bf Программирование на языке \langcpp\protect\\Лекция
11\protect\vspace{1em}\\}Многопоточное программирование}

\begin{document}
\begin{frame} 
  \titlepage
\end{frame}

\begin{frame}[fragile]{Асинхронное выполнение}
Предположим, что мы хотим вычислить \texttt{doAsyncWork} асинхронно.
\begin{lstlisting}
int doAsyncWork();
\end{lstlisting}

В \langcpp есть два способа выполнения задач асинхронно:\medskip

\fakeitem создать поток вручную \texttt{std::thread},
   
\begin{lstlisting}
#include <thread>
// создание потока, вычисляющего doAsyncWork()
std::thread t(doAsyncWork);
\end{lstlisting}

\fakeitem использование \texttt{std::async}.
\begin{lstlisting}
#include <future>
// использование std::async
std::future<int> fut = std::async(doAsyncWork);
int res = fut.get();
\end{lstlisting}

\texttt{std::async} может в некоторых случаях \\(зависит от планировщика) отложить выполнение\\ задачи до вызова \texttt{get} или \texttt{wait}.
\end{frame}

\begin{frame}[fragile]{\texttt{std::async}}
    \begin{itemize}
        \item Имеет две стратегии выполнения: асинхронное выполнение и
            отложенное (синхронное) выполнение.
                \begin{enumerate}
                    \item \texttt{std::launch::async}
                    \item \texttt{std::launch::deferred}
                \end{enumerate}
        \item По умолчанию имеет стратегию:\\
             \texttt{std::launch::async | std::launch::deferred}
\end{itemize}
\begin{lstlisting}
// гарантирует асинхронное выполнение
std::future<int> fut = 
    std::async(std::launch::async, doAsyncWork);
int res = fut.get();
\end{lstlisting}
\begin{itemize}
        \item Отложенная задача может никогда не выполниться,\\
            если не будет вызвано \texttt{get} или \texttt{wait}.

        \item Возвращает \texttt{std::future<T>}, который\\ позволяет
            получить возвращаемое значение.
            
        \item Позволяет обрабатывать исключения.
    \end{itemize}
\end{frame}

\begin{frame}[fragile]{\texttt{std::thread}}
    \begin{itemize}
        \item Сразу же начинает вычислять переданную функцию.

        \item Игнорирует возвращаемое значение функции.
    \end{itemize}
\begin{lstlisting}
// переменная для возвращаемого значения
int res = 0;
std::thread t([&res](){res = doAsyncWork();});
t.join();
\end{lstlisting}    
       
    \begin{itemize}
        \item Метод \texttt{join()} позволяет заблокировать текущий поток,\\ пока
            выполнение потока не завершится.

        \item Метод \texttt{detach()} позволяет отключить поток от объекта,\\
            т.е. разорвать связь между объектом и потоком.

        \item При вызове деструктора подключаемого потока программа
            завершается, т.е. необходимо вызвать \texttt{join} или \texttt{detach}.

                   
        \item Исключения не могут покидать пределы потока.
        
        
        \item \texttt{native\_handle()} возвращает дескриптор потока.
    \end{itemize}


\end{frame}

\begin{frame}[fragile]{Синхронизация}
    \begin{lstlisting}
double shared = 0;	// разделяемая переменная
std::mutex mtx; 	// мьютекс для shared

void compute(int begin, int end)  {
    for (int i = begin; i != end; ++i) {
        double current = someFunction(i);        
        // критическая секция
        std::lock_guard<std::mutex> lck(mtx);
        shared += current;        
    }
}

int main () {
  std::thread th1 (compute, 0,   100);
  std::thread th2 (compute, 100, 200);
  th1.join();
  th2.join();

  std::cout << shared << std::endl;
}
\end{lstlisting}
\end{frame}

\begin{frame}[fragile]{\texttt{std::atomic}}
    \begin{itemize}
        \item Шаблон \texttt{std::atomic} позволяет определить переменную,
            операции с которой будут атомарны.

        \item Определён только для целочисленных встроенных типов и указателей.
    \end{itemize}

\begin{lstlisting}
template<class T>
struct shared_ptr_data 
{
    void addref() 
    {
        ++counter; // atomic increment
    }
    
    T * ptr;
    std::atomic<size_t> counter;
};
\end{lstlisting}
\end{frame}

\begin{frame}{Общие советы и замечания}
    \begin{itemize}
        \pitem Предпочитайте \texttt{std::async} прямому созданию потоков.

        \pitem Гарантируйте неподключённость потоков на всех путях выполнения (в т.ч. при возникновении исключений).

        \pitem При использовании \texttt{std::thread} следите за тем,\\ чтобы исключения не покидали функцию потока.

        \pitem Используйте \texttt{std::atomic} вместо мьютекса,\\
            когда синхронизация нужна только для одной\\ целочисленной
            переменной.
            
        \pitem Делайте константные методы безопасными\\ в смысле потоков
            (например, при кешировании).
            
        \pitem \code{volatile}~--- это не про многопоточность.
    \end{itemize}
\end{frame}

\end{document}
