\documentclass{beamer}

\usepackage{mooc}

\usetikzlibrary{arrows, decorations.markings, shapes}
\usetikzlibrary{positioning}

% The face style, can be changed

\title{{\bf Программирование на языке \langcpp\protect\\Лекция
1\protect\vspace{1em}\\}Структура кода на \langcpp}

\begin{document}
\begin{frame} 
  \titlepage
\end{frame}

\begin{frame}[fragile]{Разбиение программы на файлы}

Зачем разбивать программу на файлы? 
\begin{itemize}
    \item С небольшими файлами удобнее работать.
    \item Разбиение на файлы структурирует код.
    \item Позволяет нескольким программистам разрабатывать 
        приложение одновременно.
    \item Ускорение повторной компиляции при небольших 
        изменениях в отдельных частях программы. 
        
\end{itemize}
    Файлы с кодом на \langcpp бывают двух типов:
    \begin{enumerate}
        \item файлы с исходным кодом (расширение \texttt{.cpp}, иногда \texttt{.C}),
        \item заголовочные файлы (расширение \texttt{.hpp} или \texttt{.h}).
    \end{enumerate}
\end{frame}

%Для того, чтобы говорить о компиляции С++ кода нужно сначала изучить
%его структуру. Программа на С++ состоит из двух типов файлов:
%файлов с исходным кодом (обычно с расширением .cpp или .С) и заголовочных файлов
%(с расширением .h или .hpp). Предполагается, что весь код написан в файлах с
%исходным кодом, а заголовочные файлы служат только для описания зависимостей.
%
%Давайте ответим на простой вопрос: зачем разбивать программу на
%файлы? Ответ просто: потому, что нам удобнее работать с небольшими файлами,
%чем с одним огромным, в котором ничего не найти. Это так же позволяет
%разрабатывать приложение по частям параллельно, когда каждый программист
%работает над какими-то своими файлами. Есть и другие причины для разделения на
%файлы -- это переиспользование кода (в случае библиотек) и ускорение перекомпиляции
%при небольших изменениях в отдельных частях программы. В таких случаях, если
%изменения затронули небольшое количество файлов, достаточно перекомпилировать
%только их.

\begin{frame}[fragile]{Заголовочные файлы}
\begin{itemize}
    \item Файл {\tt foo.cpp}:
{\small \begin{lstlisting}
// определение (definition) функции foo
void foo()
{
    bar();
}
\end{lstlisting}}

\item Файл {\tt bar.cpp}:
\begin{lstlisting}
// определение (definition) функции bar
void bar() { }
\end{lstlisting}

Компиляция этих файлов выдаст ошибку.
\end{itemize}
\end{frame}

\begin{frame}[fragile]{Заголовочные файлы}
\begin{itemize}
    \item Файл {\tt foo.cpp}:
{\small \begin{lstlisting}
// объявление (declaration) функции bar
void bar();

// определение (definition) функции foo
void foo()
{
    bar();
}
\end{lstlisting}}

\item Файл {\tt bar.cpp}:
\begin{lstlisting}
// определение (definition) функции bar
void bar() { }
\end{lstlisting}
\end{itemize}
\end{frame}

\begin{frame}[fragile]{Заголовочные файлы}
    Предположим, что мы изменили функцию {\tt bar}.
\begin{itemize}
    \item Файл {\tt foo.cpp}:
{\small \begin{lstlisting}
void bar();

void foo()
{
    bar();
}
\end{lstlisting}}

\item Файл {\tt bar.cpp}:
\begin{lstlisting}
int bar() { return 1; }
\end{lstlisting}

Данный код некорректен — объявление отличается от определения.
(Неопределённое поведение.)
\end{itemize}
\end{frame}

\begin{frame}[fragile]{Заголовочные файлы}
    Добавим заголовочный файл {\tt bar.hpp}.
\begin{itemize}

    \item Файл {\tt foo.cpp}:
\begin{lstlisting}
#include "bar.hpp"

void foo()
{
    bar();
}
\end{lstlisting}

\item Файл {\tt bar.cpp}:
\begin{lstlisting}
int bar() { return 1; }
\end{lstlisting}

    \item Файл {\tt bar.hpp}:
\begin{lstlisting}
int bar();
\end{lstlisting}
\end{itemize}
\end{frame}

\begin{frame}[fragile]{Двойное включение}
Может случиться двойное включение заголовочного файла.
\begin{itemize}
    \item Файл {\tt foo.cpp}:
\begin{lstlisting}
#include "foo.hpp"
#include "bar.hpp"

void foo()
{
    bar();
}
\end{lstlisting}

    \item Файл {\tt foo.hpp}:
\begin{lstlisting}
#include "bar.hpp"

void foo();
\end{lstlisting}
\end{itemize}
\end{frame}

\begin{frame}[fragile]{Стражи включения}
Это можно исправить двумя способами:
\begin{itemize}
    \item (наиболее переносимо) Файл {\tt bar.hpp}:
\begin{lstlisting}
#ifndef BAR_HPP
#define BAR_HPP

int bar();
#endif
\end{lstlisting}
    \item (наиболее просто) Файл {\tt bar.hpp}: 
\begin{lstlisting}
#pragma once

int bar();
\end{lstlisting}
\end{itemize}
{\bf Резюме:} {\tt .cpp} — для определений, {\tt .hpp} — для объявлений.
\end{frame}
\end{document}
