\documentclass{beamer}

\usepackage{mooc}

\usetikzlibrary{arrows, decorations.markings, shapes}
\usetikzlibrary{positioning}

% The face style, can be changed

\title{{\bf Программирование на языке \langcpp\protect\\Лекция
4\protect\vspace{1em}\\}Перегрузка}

\begin{document}
\begin{frame} 
  \titlepage
\end{frame}

\begin{frame}[fragile]{Перегрузка функций}
    В отличие от \langc в \langcpp можно определить несколько функций
    с одинаковым именем, но разными параметрами.
    \begin{lstlisting}
double square(double d) { return d * d; }

int    square(int    i) { return i * i; }
    \end{lstlisting}
    При вызове функции по имени будет произведен поиск наиболее подходящей
    функции:
\begin{lstlisting}
int     a = square(4);      // square(int)
double  b = square(3.14);   // square(double)
double  c = square(5);      // square(int)
int     d = square(b);      // square(double)
float   e = square(2.71f);  // square(double)
\end{lstlisting}
\end{frame}

    
\begin{frame}[fragile]{Перегрузка методов}
    \begin{lstlisting}
struct Vector2D {
    Vector2D(double x, double y) : x(x), y(y) {}

    Vector2D mult(double d) const 
        { return Vector2D(x * d, y * d); }
    
    double  mult(Vector2D const& p) const 
        { return x * p.x + y * p.y;   }

    double x, y;
};
\end{lstlisting}

\begin{lstlisting}
    Vector2D p(1, 2);
    Vector2D q = p.mult(10); // (10, 20)
    double  r  = p.mult(q);  // 50 
\end{lstlisting}

\end{frame}

\begin{frame}[fragile]{Перегрузка при наследовании}
    \begin{lstlisting}
struct File {
    void write(char const * s);
    ...
};

struct FormattedFile : File {
    void write(int i);
    void write(double d);
    using File::write;
    ...
};
    \end{lstlisting}

    \begin{lstlisting}
    FormattedFile f;
    f.write(4);
    f.write("Hello");
    \end{lstlisting}

\end{frame}


\begin{frame}[fragile]{Правила перегрузки}
    \begin{enumerate}
        \item Если есть точное совпадение, то используется оно.

        \item Если нет функции, которая могла бы подойти с учётом преобразований,
            выдаётся ошибка.

        \item Есть функции, подходящие с учётом преобразований:

            \begin{enumerate}
                \item Расширение типов.\\
                    \code{char, signed char, short} $\to$ \code{int}\\
                    \code{unsigned char, unsigned short} $\to$ \code{int}/\code{unsigned int}\\
                    \code{float} $\to$ \code{double}

                \item Стандартные преобразования (числа, указатели).
                \item Пользовательские преобразования.
            \end{enumerate}
            В случае нескольких параметров нужно, чтобы выбранная функция была
            {\em строго лучше} остальных.
    \end{enumerate}
    {\bf NB:} перегрузка выполняется на этапе компиляции.
\end{frame}
\end{document}

 
