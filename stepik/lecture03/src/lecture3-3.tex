\documentclass{beamer}

\usepackage{mooc}

\usetikzlibrary{arrows, decorations.markings, shapes}
\usetikzlibrary{positioning}

% The face style, can be changed

\title{{\bf Программирование на языке \langcpp\protect\\Лекция
3\protect\vspace{1em}\\}Конструкторы и деструкторы}

\begin{document}
\begin{frame} 
  \titlepage
\end{frame}

\begin{frame}[fragile]{Конструкторы}
    Конструкторы — это методы для инициализации структур.
\begin{lstlisting}
struct Point {
    Point() { 
        x = y = 0;
    }
    Point(double x, double y) {
        this->x = x;
        this->y = y;
    }
    double x;
    double y;
};
\end{lstlisting}
\begin{lstlisting}
    Point p1;
    Point p2(3,7);
\end{lstlisting}
\end{frame}

\begin{frame}[fragile]{Список инициализации}
    Список инициализации позволяет проинициализировать поля до входа в
    конструктор.
\begin{lstlisting}
struct Point {
    Point() : x(0), y(0) 
    {}
    Point(double x, double y) : x(x), y(y) 
    {}

    double x;
    double y;
};
\end{lstlisting}
Инициализации полей в списке инициализации
происходит в {\em порядке объявления полей} в структуре.
\end{frame}

\begin{frame}[fragile]{Значения по умолчанию}
    \begin{itemize}
        \item Функции могут иметь значения параметров {\em по умолчанию}.
        \item Значения параметров по умолчанию нужно указывать в {\em объявлении
            функции}.
\begin{lstlisting}
struct Point {
    Point(double x = 0, double y = 0) 
        : x(x), y(y)
    {}
    double x;
    double y;
};
\end{lstlisting}
\begin{lstlisting}
    Point p1;
    Point p2(2);
    Point p3(3,4);
\end{lstlisting}
    \end{itemize}
\end{frame}

\begin{frame}[fragile]{Конструкторы от одного параметра}
    Конструкторы от одного параметра задают {\em неявное}
    пользовательское преобразование:
\begin{lstlisting}
struct Segment {
    Segment() {}
    Segment(double length) 
        : p2(length, 0)
    {}
    Point p1;
    Point p2;
};
\end{lstlisting}
\begin{lstlisting}
    Segment s1;
    Segment s2(10);
    Segment s3 = 20;
\end{lstlisting}
\end{frame}

\begin{frame}[fragile]{Конструкторы от одного параметра}
    Для того, чтобы запретить {\em неявное} пользовательское
преобразование, используется ключевое слово \code{explicit}.
\begin{lstlisting}
struct Segment {
    Segment() {}
    explicit Segment(double length) 
        : p2(length, 0)
    {}
    Point p1;
    Point p2;
};
\end{lstlisting}
\begin{lstlisting}
    Segment s1;
    Segment s2(10);
    Segment s3 = 20; // error
\end{lstlisting}
\end{frame}

\begin{frame}[fragile]{Конструкторы от одного параметра}
Неявное пользовательское преобразование, 
задаётся также конструкторами, которые могут принимать один параметр.
\begin{lstlisting}
struct Point {
    explicit Point(double x = 0, double y = 0) 
        : x(x), y(y)
    {}
    double x;
    double y;
};
\end{lstlisting}
\begin{lstlisting}
    Point p1;
    Point p2(2);
    Point p3(3,4);
    Point p4 = 5; // error
\end{lstlisting}
\end{frame}

\begin{frame}[fragile]{Конструктор по умолчанию}
    Если у структуры нет конструкторов, то конструктор 
    без параметров, {\em конструктор по умолчанию},
    генерируется компилятором.
\begin{lstlisting}
struct Segment {
    Segment(Point p1, Point p2) 
        : p1(p1), p2(p2)
    {}
    Point p1;
    Point p2;
};
\end{lstlisting}
\begin{lstlisting}
    Segment s1; // error
    Segment s2(Point(), Point(2,1));
\end{lstlisting}
\end{frame}

\begin{frame}[fragile]{Особенности синтаксиса \langcpp}

    {\em ``Если что-то похоже на объявление функции, то это
    и есть объявление функции.''}
\begin{lstlisting}
struct Point {
    explicit Point(double x = 0, double y = 0) 
        : x(x), y(y) {}
    double x;
    double y;
};
\end{lstlisting}
\begin{lstlisting}
    Point p1;   // определение переменной
    Point p2(); // объявление функции

    double k = 5.1;
    Point p3(int(k)); // объявление функции
    Point p4((int)k); // определение переменной
\end{lstlisting}
\end{frame}

\begin{frame}[fragile]{Деструктор}
    Деструктор — это метод, который вызывается при удалении структуры,
    генерируется компилятором.
\begin{lstlisting}
struct IntArray {
    explicit IntArray(size_t size)
        : size(size)
        , data(new int[size])
    { }

    ~IntArray() {
        delete [] data;
    }
    
    size_t size;
    int *  data;
};
\end{lstlisting}
\end{frame}

\begin{frame}[fragile]{Время жизни}{}
    {\em Время жизни}~--- это временной интервал 
    между вызовами конструктора и деструктора.
\begin{lstlisting}
void foo()
{
    IntArray a1(10); // создание a1
    IntArray a2(20); // создание a2 
    for (size_t i = 0; i != a1.size; ++i) {
        IntArray a3(30); // создание a3
        ...
    } // удаление a3
} // удаление a2, потом a1
\end{lstlisting}
Деструкторы переменных на стеке вызываются в обратном
порядке (по отношению к порядку вызова конструкторов).
\end{frame}

\end{document}
