\documentclass[aspectration=1610,t]{beamer}

\usepackage{mooc}

\title{{\bf Программирование на языке \langcpp\protect\\Лекция
8\protect\vspace{1em}\\}Стандарты \langcpp[11]/\langcpp[14]}

\begin{document}
\begin{frame} 
  \titlepage
\end{frame}

\begin{frame}[fragile]{Стандартизация \langcpp}
    \begin{itemize}
        \pause\item[1983] Появление \langcpp.
        \pause\item[1998] Первый стандарт ISO/IEC 14882:1998.
        \pause\item[2003] Стандарт ISO/IEC 14882:2003, исправляющий недостатки стандарта \langcpp[98].
        \pause\item[2011] Стандарт ISO/IEC 14882:2011.
        \pause\item[2014] Стандарт ISO/IEC 14882:2014, исправляющий недостатки стандарта \langcpp[11].
        \pause\item[2017] К концу года планируется выход нового стандарта.
    \end{itemize}
\end{frame}

\begin{frame}[fragile]{Основные принципы разработки стандарта}
    \begin{itemize}
        \pause\item поддержка совместимости с предыдущими стандартами;
        \pause\item улучшение техники программирования;
        \pause\item улучшение \langcpp с точки зрения дизайна; 
        \pause\item увеличение типобезопасности для обеспечения безопасной альтернативы для существующих опасных подходов;
        \pause\item увеличение производительности;
        \pause\item «не платить за то, что не используешь»;
        \pause\item введение новых возможностей через стандартную\\ библиотеку, а~не через ядро языка;                
        \pause\item сделать \langcpp проще для изучения\\
        (сохраняя возможности, используемые программистами-экспертами).
    \end{itemize}
\end{frame}

\begin{frame}[fragile]{Мелкие улучшения}
\begin{enumerate}
    \pitem Исправлена проблема с угловыми скобками: {\tt T<U<int{>}>}.

    \pitem Определены понятия ``тривиальный класс'' и ``класс со стандартным размещением''.
    
    \pitem Ключевое слово \code{explicit} для оператора приведения типа.
\begin{lstlisting}
explicit operator bool () { ... }
\end{lstlisting}

    \pitem Шаблонный \code{typedef}
\begin{lstlisting}
template<class A, class B, int N> 
class SomeType;

template<typename B>
using TypedefName = SomeType<double, B, 5>;
\end{lstlisting}
\pause
\begin{lstlisting}
typedef void (*OtherType)(double); 
using OtherType = void (*)(double);
\end{lstlisting}
\end{enumerate}
\end{frame}


\begin{frame}[fragile]{Мелкие улучшения (продолжение)}
\begin{enumerate} \setcounter{enumi}{4}
    \pitem Добавлен тип \code{long long int}.

    \pitem Добавлена библиотека поддержки типов: по типу на этапе компиляции можно узнавать его свойства\\ (см. заголовочный файл \texttt{<type\_traits>}).

    \pitem Добавлены операторы \code{alignof} и \code{alignas}.
\begin{lstlisting}
alignas(float) unsigned char c[sizeof(float)];
\end{lstlisting}

    \pitem Добавлен \code{static\_assert}
\begin{lstlisting}
template <class T>
void run(T * data, size_t n) {
   static_assert(std::is_signed<T>::value, 
                 "T is not signed.");
}
\end{lstlisting}

%\item Внешние шаблоны.
%\begin{lstlisting}
%//instantiate here
%template class std::vector<MyClass>; 
%
%//don't instantiate here
%extern template class std::vector<MyClass>; 
%\end{lstlisting}
%    \item \code{sizeof} без содания экземпляра
%\begin{lstlisting}
%struct SomeType { OtherType member; };
% 
%sizeof(SomeType::member); //only C++11
%\end{lstlisting}
\end{enumerate}
\end{frame}

\begin{frame}[fragile]{\code{nullptr}}
    В язык добавлены тип \code{std::nullptr\_t} и литерал \code{nullptr}.

\begin{lstlisting}          
void foo(int a)     { ... }

void foo(int * p)   { ... }

void bar() 
{
    foo(0); // вызов foo(int a)
    foo((int *) 0); // \langcpp[98]
    foo(nullptr);   // \langcpp[11]
}
\end{lstlisting}
\medskip
Тип \code{std::nullptr\_t} имеет единственное \\значение \code{nullptr},
которое неявно приводится\\ к нулевому указателю на любой тип.
\end{frame}

\begin{frame}[fragile]{Вывод типов}
\begin{lstlisting}
Array<Unit *> units;

for(size_t i = 0; i != units.size(); ++i) {
    // Unit *
    auto u = units[i]; 	    
    
    // Array<Item> const \& 
    decltype(u->items()) items = u->items();
    ...
\end{lstlisting}
\pause
\begin{lstlisting}
    auto a = items[0];        // a - Item  
    decltype(items[0]) b = a; // b - Item const \& 
    
    decltype(a)    c = a;     // c - Item
    decltype((a))  d = a;     // d - Item \&

    decltype(b)    e = b;     // e - Item const \&
    decltype((b))  f = b;     // f - Item const \&
\end{lstlisting}
\end{frame}

\begin{frame}[fragile]{Альтернативный синтаксис для функций}
\begin{lstlisting}
// RETURN\_TYPE = ?
template <typename A, typename B> 
RETURN_TYPE Plus(A a, B b) { return a + b; }
\end{lstlisting}
\begin{lstlisting}
// некорректно, a и b определены позже
template <typename A, typename B> 
decltype(a + b) Plus(A a, B b) { return a + b; }
\end{lstlisting}
%\begin{lstlisting}
%template <typename A, typename B> 
%    decltype(std::declval<const A&>() + std::declval<const B&>())
%        Plus(const A &a, const B &b) {
%    return a + b;
%}
%
\begin{lstlisting}
// \langcpp[11]
template <typename A, typename B> 
auto Plus(A a, B b) -> decltype(a + b) { 
    return a + b; 
}
\end{lstlisting}
\begin{lstlisting}
// \langcpp[14]
template <typename A, typename B> 
auto Plus(A a, B b) {
    return a + b;
}
\end{lstlisting}
\end{frame}

\begin{frame}[fragile]{Шаблоны с переменным числом аргументов}
    \begin{lstlisting}
void printf(char const *s) {
    while (*s) {
        if (*s == '%' && *(++s) != '%')
            // обработать ошибку
        std::cout << *s++;
    }
}
template<typename T, typename... Args>
void printf(char const *s, T value, Args... args) {
    while (*s) {
        if (*s == '%' && *(++s) != '%') {
            std::cout << value;
            printf(++s, args...);
            return;
        }
        std::cout << *s++;
    }
    // обработать ошибку
}
\end{lstlisting}
\end{frame}

% TODO: сделать текстовый степ
%\begin{frame}[fragile]{Наследование от списка типов}
%\begin{lstlisting}
%template <typename... BaseClasses> 
%struct ClassName : BaseClasses... {
%    ClassName (BaseClasses&&... base_classes) 
%        : BaseClasses(base_classes)... 
%    {}
%};
%\end{lstlisting}
%
%Можно использовать \code{sizeof...} для определения количества %параметров. 
%\begin{lstlisting}
%template<typename ...Args> struct SomeStruct {
%    static const int count = sizeof...(Args);
%};
%\end{lstlisting}
%\end{frame}


\begin{frame}[fragile]{Ключевые слова \code{default} и \code{delete}}
\begin{lstlisting}
struct SomeType {
    SomeType() = default; // Конструктор по умолчанию.
    SomeType(OtherType value);
};

struct NonCopyable {
    NonCopyable() = default;
    NonCopyable(const NonCopyable&) = delete;
    NonCopyable & operator=(const NonCopyable&) = delete;
};
\end{lstlisting}
\pause\medskip
Удалять можно и обычные функции.
\begin{lstlisting}
template<class T>
void foo(T const * p) { ... }

void foo(char const *) = delete;
\end{lstlisting}
\end{frame}

\begin{frame}[fragile]{Делегация конструкторов}
    \begin{lstlisting}
struct SomeType  {
    SomeType(int newNumber): number(newNumber) {}
    SomeType() : SomeType(42) {}
private:
    int number;
};
struct SomeClass {
    SomeClass() {}
    explicit SomeClass(int newValue): value(newValue) {}
private:
    int value = 5;
};
struct BaseClass {
    BaseClass(int value);
};
struct DerivedClass : public BaseClass {
    using BaseClass::BaseClass;
};
\end{lstlisting}
\end{frame}


\begin{frame}[fragile]{Явное переопределение и финальность}
    \begin{lstlisting}
struct Base {
    virtual void update();
    virtual void foo(int);
    virtual void bar() const;
};
struct Derived : Base {
    void updata() override;          	  // error
    void foo(int) override;               // OK
    virtual void foo(long) override;      // error
    virtual void foo(int) const override; // error
    virtual int  foo(int) override;       // error
    virtual void bar(long);               // OK
    virtual void bar() const final;       // OK
};
struct Derived2 final : Derived {
    virtual void bar() const;       // error
};
struct Derived3 : Derived2 {};      // error
    \end{lstlisting}
\end{frame}

% TODO: слайд про enum-ы
%    \item Перечисления со строгой типизацией: 
%\begin{lstlisting}
%enum class Enum1 { Val1, Val2, Val3 = 100, Val4 };
%enum class Enum2 : unsigned int { Val1, Val2i };
%\end{lstlisting}

\end{document}