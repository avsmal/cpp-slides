\documentclass{beamer}

\usepackage{mooc}

\usetikzlibrary{arrows, decorations.markings, shapes}
\usetikzlibrary{positioning}

% The face style, can be changed

\title{{\bf Программирование на языке \langcpp\protect\\Лекция
2\protect\vspace{1em}\\}Строки и ввод-вывод}

\begin{document}
\begin{frame} 
  \titlepage
\end{frame}

\begin{frame}[fragile]{Строковые литералы}
    \begin{itemize}
        \item Строки~--- это массивы символов
            типа \code{char}, заканчивающиеся нулевым символом.
\begin{lstlisting}
// массив 'H', 'e', 'l', 'l', 'o', '\verb!\!0'
char s[] = "Hello";

\end{lstlisting}

        \item Строки могут содержать управляющие последовательности:
            \begin{enumerate}
                \item \verb!\n!~--- перевод строки,
                \item \verb!\t!~--- символ табуляции,
                \item \verb!\\!~--- символ '\verb!\!',
                \item \verb!\"!~--- символ '\verb!"!',
                \item \verb!\0!~--- нулевой символ.
            \end{enumerate}

\begin{lstlisting}
cout << "List:\n\t- C,\n\t- C++.\n";
\end{lstlisting}
    \end{itemize}
\end{frame}

\begin{frame}[fragile]{Работа со строками в стиле \langc}
    \begin{itemize}
        \item Библиотека \code{cstring} предлагает множество
            функций для работы со строками (\code{char *}).
\begin{lstlisting}
char s1[100] = "Hello";
cout << strlen(s1) << endl; // 5

char s2[] = ", world!";
strcat(s1, s2);
                               
char s3[6] = {72, 101, 108, 108, 111};
if (strcmp(s1, s3) == 0)
    cout << "s1 == s3" << endl;
\end{lstlisting}
        \item Работа со строками в стиле \langc предполагает
            кропотливую работу с ручным выделением памяти.
    \end{itemize}
\end{frame}

\begin{frame}[fragile]{Работа со строками в стиле \langcpp}
        Библиотека \code{string} предлагает обёртку
            над строками, которая позволяет упростить все
            операции со строками.

\small
\begin{lstlisting}
#include <string>
using namespace std;

int main() {
    string s1 = "Hello";
    cout << s1.size() << endl; // 5

    string s2 = ", world!";
    s1 = s1 + s2;
                               
    if (s1 == s2)
        cout << "s1 == s2" << endl;
    return 0;
}
\end{lstlisting}
\end{frame}

\begin{frame}[fragile]{Ввод-вывод в стиле \langc}
    \begin{itemize}
        \item Библиотека \code{cstdio} предлагает функции
            для работы со стандартным вводом-выводом.

        \item Для вывода используется функция {\tt printf}:
\begin{lstlisting}
#include <cstdio>

int main() {
    int h = 20, m = 14;
    printf("Time: %d:%d\n", h, m);
    printf("It's %.2f hours to midnight\n", 
            ((24 - h) * 60.0 - m) / 60);
    return 0;
}
\end{lstlisting}
    \end{itemize}
\end{frame}

\begin{frame}[fragile]{Ввод-вывод в стиле \langc}
    \begin{itemize}
        \item Библиотека \code{cstdio} предлагает функции
            для работы со стандартным вводом-выводом.

        \item Для ввода используется функция {\tt scanf}:
\begin{lstlisting}
#include <cstdio>

int main() {
    int a = 0, b = 0;
    printf("Enter a and b: ");
    scanf("%d %d", &a, &b);
    printf("a + b = %d\n", (a + b));
    return 0;
}
\end{lstlisting}
    \item Ввод-вывод в стиле \langc{} достаточно сложен и
        небезопасен (типы аргументов не проверяются).
    \end{itemize}
\end{frame}

\begin{frame}[fragile]{Ввод-вывод в стиле \langcpp}
\begin{itemize}    
    \item В \langcpp ввод-вывод реализуется через библиотеку \code{iostream}.
    \begin{lstlisting}
#include <string>
#include <iostream>
using namespace std;

int main() {
    string name;
    cout << "Enter your name: ";
    cin >> name; // считывается слово
    cout << "Hi, " << name << endl;

    return 0;
}
    \end{lstlisting}
    \item Реализация ввода-вывода в стиле \langcpp типобезопасна.
\end{itemize}
\end{frame}

\begin{frame}[fragile]{Работа с файлами в стиле \langcpp}
\begin{itemize}    
    \small
    \item Библиотека \code{fstream} обеспечивает работу с файлами.
    \begin{lstlisting}
#include <string>
#include <fstream>
using namespace std;

int main() {
    string name;
    ifstream input("input.txt");
    input >> name;

    ofstream output("output.txt");
    output << "Hi, " << name << endl;
    return 0;
}
    \end{lstlisting}
    \item Файлы закроются при выходе из функции.
\end{itemize}
\end{frame}


\end{document}
