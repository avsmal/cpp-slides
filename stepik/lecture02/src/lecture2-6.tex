\documentclass{beamer}

\usepackage{mooc}

\usetikzlibrary{arrows, decorations.markings, shapes}
\usetikzlibrary{positioning}

% The face style, can be changed

\title{{\bf Программирование на языке \langcpp\protect\\Лекция
2\protect\vspace{1em}\\}Динамическая память}

\begin{document}
\begin{frame} 
  \titlepage
\end{frame}

%Использование памяти в работающей программе.
%Стек и куча.

\begin{frame}[fragile]{Зачем нужна динамическая память?}
    \begin{itemize}
        \item Стек программы ограничен. Он не предназначен
            для хранения больших объемов данных.
\begin{lstlisting}
// Не умещается на стек
double m[10000000] = {}; // 80 Mb
\end{lstlisting}

        \item Время жизни локальных переменных ограничено
            временем работы функции.

        \item Динамическая память выделяется в сегменте данных.

        \item Структура, отвечающая за выделение дополнительной
            памяти, называется {\bf кучей} (не нужно путать
            с одноимённой структурой данных).

        \item Выделение и освобождение памяти {\em управляется вручную}.
    \end{itemize}
\end{frame}

\begin{frame}[fragile]{Выделение памяти в стиле \langc}
    \begin{itemize}
        \item Стандартная библиотека \code{cstdlib}
            предоставляет четыре функции для управления памятью:
            {\small
\begin{lstlisting}
void * malloc (size_t size);
void   free   (void * ptr);
void * calloc (size_t nmemb, size_t size);
void * realloc(void * ptr, size_t size);
\end{lstlisting}
}
        \item \code{size\_t}~--- специальный целочисленный беззнаковый тип, 
            может вместить в себя размер любого типа в байтах.

        \item Тип \code{size\_t} используется для указания размеров типов данных,
            для индексации массивов и пр.

        \item \code{void *}~--- это указатель на нетипизированную память\\
            (раньше для этого использовалось \code{char *}).

    \end{itemize}
\end{frame}

\begin{frame}[fragile]{Выделение памяти в стиле \langc}
    \begin{itemize}
        \item Функции для управления памятью в стиле \langc:
            {\small
\begin{lstlisting}
void * malloc (size_t size);
void * calloc (size_t nmemb, size_t size);
void * realloc(void * ptr, size_t size);
void   free   (void * ptr);               
\end{lstlisting}
}
        \item \code{malloc}~--- выделяет область памяти размера $\ge$
            \code{size}. Данные не инициализируются.

        \item \code{calloc}~--- выделяет массив из \code{nmemb} размера 
            \code{size}. Данные инициализируются нулём.

        \item \code{realloc}~--- изменяет размер области памяти по указателю
            \code{ptr} на \code{size} (если возможно, то это делается на месте).

        \item \code{free}~--- освобождает область памяти, ранее выделенную
            одной из функций \code{malloc}/\code{calloc}/\code{realloc}.
            
    \end{itemize}
\end{frame}

\begin{frame}[fragile]{Выделение памяти в стиле \langc}
    \begin{itemize}
        \item Для указания размера типа используется оператор \code{sizeof}.
\begin{lstlisting}
// создание массива из 1000 int    
int * m = (int *)malloc(1000 * sizeof(int));
m[10] = 10;

// изменение размера массива до 2000
m = (int *)realloc(m, 2000 * sizeof(int));

// освобождение массива
free(m);

// создание массива нулей
m = (int *)calloc(3000, sizeof(int));

free(m);
m = 0;
\end{lstlisting}
    \end{itemize}
\end{frame}


\begin{frame}[fragile]{Выделение памяти в стиле \langcpp}
    \begin{itemize}
        \item Язык \langcpp предоставляет два набора операторов для
            выделения памяти:
            \begin{enumerate}
                \item \code{new} и \code{delete}~--- для одиночных значений,

                \item \code{new []} и \code{delete []}~--- для массивов.
            \end{enumerate}

        \item Версия оператора \code{delete} должна соответствовать
            версии оператора \code{new}.
\begin{lstlisting}
// выделение памяти под один int со значением 5
int * m = new int(5);
delete m; // освобождение памяти

// создание массива значений типа \code{int}
m = new int[1000];
delete [] m; // освобождение памяти
\end{lstlisting}
    \end{itemize}
\end{frame}

\begin{frame}[fragile]{Типичные проблемы при работе с памятью}
    \begin{itemize}
        \item Проблемы производительности: создание переменной на стеке
            намного ``дешевле'' выделения для неё динамической памяти.

        \item Проблема фрагментации: выделение большого количества
            небольших сегментов способствует фрагментации памяти.
        
        \item Утечки памяти:
\begin{lstlisting}
// создание массива из 1000 int    
int * m = new int[1000];

// создание массива из 2000 int    
m = new int[2000]; // утечка памяти

// Не вызван delete [] m, утечка памяти
\end{lstlisting}
    \end{itemize}
\end{frame}

\begin{frame}[fragile]{Типичные проблемы при работе с памятью}
    \begin{itemize}
    \item Неправильное освобождение памяти.
\begin{lstlisting}
int * m1 = new int[1000];
delete m1; // должно быть delete [] m1

int * p = new int(0);
free(p); // совмещение функций C++ и C

int * q1 = (int *)malloc(sizeof(int));
free(q1);
free(q1); // двойное удаление

int * q2 = (int *)malloc(sizeof(int));
free(q2);
q2 = 0;   // обнуляем указатель
free(q2); // правильно работает для q2 = 0
\end{lstlisting}
\end{itemize}
\end{frame}

\end{document}


\end{document}
