\documentclass{beamer}
\usepackage{csc}
\title{STL: последовательные контейнеры}

\date{
   \textbf{CS центр}\\
   13 марта 2018 \\
   Санкт-Петербург
}

\lstset{basicstyle=\fontsize{8pt}{0.92em}\ttfamily,  
        keywordstyle=\fontsize{8pt}{0.92em}\color{MOOCBlue}\bfseries, 
        commentstyle=\fontsize{8pt}{0.92em}\ttfamily\color{MOOCGreen}}

\begin{document}
\begin{frame} 
  \titlepage
\end{frame}

\begin{frame}{STL: введение}
    \begin{itemize}
        \item STL = Standard Template Library.
        
        \item STL является частью стандартной библиотеки \langcpp, описана в стандарте, но не упоминается там явно.
            
        \item Авторы: Александр Степанов, Дэвид Муссер и Менг Ли\\ 
            (сначала для HP, а потом для SGI).

        \item Основана на разработках для языка Ада.
    \end{itemize}
\pause
\begin{block}{Основные составляющие}
    \begin{itemize}
        \item контейнеры (хранение объектов в памяти),
        \item итераторы (доступ к элементам контейнера),
        \item адаптеры (обёртки над контейнерами),
        \item алгоритмы (для работы с последовательностями),
        \item функциональные объекты, функторы (обобщение функций).
    \end{itemize}
\end{block}
\end{frame}

\begin{frame}{Преимущества стандартной библиотеки}{}
    \begin{itemize}
        \item стандартизированность,
        \item общедоступность,
        \item эффективность,
        \item общеизвестность,
        \item \dots
    \end{itemize}
\pause
\begin{block}{Чего нет в стандартной библиотеке?}
    \begin{itemize}
        \item сложных структур данных,
        \item сложных алгоритмов,
        \item работы с графикой/звуком,
        \item ...
    \end{itemize}
\end{block}
\end{frame}

\begin{frame}{Общие сведения о контейнерах}
\small
    Контейнеры библиотеки STL можно разделить на две категории: 
    \begin{itemize}
        \item последовательные, 
        \item ассоциативные.
    \end{itemize}
    \pause
    \begin{block}{Общие вложенные типы}
    \begin{itemize} \tt
        \item Container::value\_type
        \item Container::iterator
        \item Container::const\_iterator 
    \end{itemize}
    \end{block}\pause\vspace{-3mm}
    \begin{block}{Общие методы контейнеров}
    \begin{itemize}
    \item Все ,,особенные методы`` и swap.

    \item \texttt{size, max\_size, empty, clear}.
    
    \item \texttt{begin, end, сbegin, сend}.

    \item Операторы сравнения: \texttt{==, !=, >, >=, <, <=}.
    \end{itemize}
\end{block}\vspace{-2mm}
\textbf{Примечание:} вся STL определена в пространстве имён \texttt{std}.
\end{frame}

\begin{frame}[fragile]{Шаблон {\tt array}}
    Класс-обёртка над статическим массивом.

\begin{itemize}
    \item \texttt{operator[], at},
    \item {\tt back, front}.
    \item \texttt{fill},
    \item \texttt{data}.
\end{itemize}
Позволяет работать с массивом как с контейнером.

\begin{lstlisting}
#include <array>

std::array<std::string, 3> a = {"One", "Two", "Three"};
std::cout << a.size() << std::endl;
std::cout << a[1] << std::endl;

// ошибка времени выполнения
std::cout << a.at(3) << std::endl; 
\end{lstlisting}
\end{frame}

\begin{frame}{Общие методы остальных последовательных контейнеров}
\begin{itemize}
    \item Конструктор от двух итераторов.  
    \item Конструктор от \texttt{count} и \texttt{defVal}.
    \item Конструктор от \texttt{std::initializer\_list<T>}.
    \item Методы {\tt back, front}.
    \item Методы \texttt{push\_back, emplace\_back}    
    \item Методы {\tt assign}.
    \item Методы {\tt insert}. 
    \item Meтоды \texttt{emplace}.
    \item Методы \texttt{erase} от одного и двух итераторов. 
\end{itemize}
\end{frame}

\begin{frame}[fragile]{Шаблон {\tt vector}}
    Динамический массив с автоматическим изменением размера\\ при
    добавлении элементов.

\begin{itemize}
    \item \texttt{operator[], at},
    \item \texttt{resize},
    \item \texttt{capacity, reserve, shrink\_to\_fit},
    \item \texttt{pop\_back},
    \item \texttt{data}.
\end{itemize}

Позволяет работать со старым кодом.
\begin{lstlisting}
#include <vector>

std::vector<std::string> v = {"One", "Two"};
v.reserve(100);
v.push_back("Three");
v.emplace_back("Four");
legacy_function(v.data(), v.size());
std::cout << v[2] << std::endl;
\end{lstlisting}
\end{frame}
           
\begin{frame}[fragile]{Шаблон {\tt deque}}
Контейнер c возможностью быстрой вставки и удаления элементов на
обоих концах за $O(1)$. Реализован как список указателей на массивы
фиксированного размера.
\begin{itemize}
    \item {\tt operator[], at},
    \item {\tt resize},
    \item {\tt push\_front, emplace\_front}
    \item \texttt{pop\_back, pop\_front},
    \item \texttt{shrink\_to\_fit}.
\end{itemize}
\begin{lstlisting}
#include <deque>

std::deque<std::string> d = {"One", "Two"};
d.emplace_back("Three");
d.emplace_front("Zero");
std::cout << d[1] << std::endl;
\end{lstlisting}

\end{frame}

\begin{frame}[fragile]{Шаблон {\tt list}}
Двусвязный список. В любом месте контейнера 
вставка и удаление производятся за $O(1)$.
\begin{itemize}к
    \item {\tt push\_front, emplace\_front},
    \item \texttt{pop\_back, pop\_front},
    \item \texttt{splice},
    \item {\tt merge, remove, remove\_if, reverse, \\sort, unique}.
\end{itemize}
\begin{lstlisting}
#include <list>

std::list<std::string> l = {"One", "Two"};
l.emplace_back("Three");
l.emplace_front("Zero");
std::cout << l.front() << std::endl;
\end{lstlisting}
\end{frame}

\begin{frame}[fragile]{Итерация по списку}
У списка нет методов для доступа к элементам по индексу.\\
Можно использовать range-based for:
\begin{lstlisting}
using std::string;
std::list<string> l = {"One", "Two", "Three"};
for (string & s : l)
    std::cout << s << std::endl;
\end{lstlisting}
Для более сложных операций нужно использовать \emph{итераторы}.
\begin{lstlisting}
std::list<string>::iterator i = l.begin();
for ( ; i != l.end(); ++i)
    if (*i == "Two")
        break;
l.erase(i);
\end{lstlisting}

Итератор списка можно перемещать в обоих направлениях:
\begin{lstlisting}    
auto last = l.end();
--last; // последний элемент
\end{lstlisting}
\end{frame}

\begin{frame}[fragile]{Шаблон \texttt{forward\_list}}
Односвязный список. В любом месте контейнера 
вставка и удаление производятся за $O(1)$.
\begin{itemize}
    \item \texttt{insert\_after} и \texttt{emplace\_after} вместо \texttt{insert} и \texttt{emplace},
    \item  \texttt{before\_begin, cbefore\_begin},
    \item {\tt push\_front, emplace\_front, pop\_front},
    \item {\tt splice\_after},
    \item {\tt merge, remove, remove\_if, reverse,\\ sort, unique}.
\end{itemize}

\begin{lstlisting}
#include <forward_list>
using std::string;

std::forward_list<string> fl = {"One", "Two"};
fl.emplace_front("Zero");
fl.push_front("Minus one");
std::cout << fl.front() << std::endl;
\end{lstlisting}
\end{frame}

\begin{frame}[fragile]{Шаблон {\tt basic\_string}}
Контейнер для хранения символьных последовательностей.
\begin{lstlisting}
typedef basic_string<char>	string;
typedef basic_string<wchar_t> 	wstring;
typedef basic_string<char16_t> 	u16string;
typedef basic_string<char32_t>	u32string;
\end{lstlisting}
\begin{itemize}
    \item Метод {c\_str()} для совместимости со старым кодом, 
    \item поддержка неявных преобразований с \langc-строками,
    \item \texttt{operator[], at},
    \item \texttt{reserve, capacity, shrink\_to\_fit},
    \item \texttt{append, operator+, operator+=},
    \item \texttt{substr, replace, compare,} 
    \item \texttt{find, rfind, find\_first\_of,\\ find\_first\_not\_of,
    find\_last\_of,\\ find\_last\_not\_of} (в терминах {\em индексов})
\end{itemize}
\end{frame}

\begin{frame}{Адаптеры и псевдоконтейнеры}
    Адаптеры:
    \begin{itemize}
        \item {\tt stack} —
            реализация интерфейса стека.

        \item {\tt queue} —
            реализация интерфейса очереди.

        \item {\tt priority\_queue} —
            очередь с приоритетом на куче.
    \end{itemize}
    
    Псевдо-контейнеры:
    \begin{itemize}
        \item {\tt vector<bool>}
        \begin{itemize}
            \item ненастоящий контейнер (не хранит {\tt bool}-ы),
            \item использует proxy-объекты.
        \end{itemize}
        \item {\tt bitset}\\
            Служит для хранения битовых масок.\\ Похож на
            {\tt vector<bool>}.
                        
        \item {\tt valarray}\\
            Шаблон служит для хранения числовых массивов
            и оптимизирован для достижения повышенной\\ вычислительной
            производительности. 
    \end{itemize}
\end{frame}

\begin{frame}[fragile]{Ещё о {\tt vector}}
\small\setlength{\baselineskip}{0.6em}
 \begin{itemize}
     \item 	Самый универсальный последовательный контейнер. 

     \item Во многих случаях самый эффективный.
         
     \item Предпочитайте \texttt{vector} другим контейнерам.
 
     \item Интерфейс построен на итераторах, а не на индексах.
 
     \item Итераторы ведут себя как указатели.
\end{itemize}

\medskip
Использование {\tt reserve} и {\tt capacity}:
    \begin{lstlisting}
std::vector<int> v;
v.reserve(N); // N — верхняя оценка на размер
...
if (v.capacity() == v.size()) // реаллокация
    \end{lstlisting}

Сжатие и очистка в \langcpp[03]:
    \begin{lstlisting}
std::vector<int> & v = getData();
// shrink\_to\_fit
std::vector<int>(v).swap(v); 
// clear + shrink\_to\_fit 
std::vector<int>().swap(v);
    \end{lstlisting}

\medskip

\end{frame}


\end{document}


