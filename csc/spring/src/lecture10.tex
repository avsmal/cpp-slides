\documentclass{beamer}
\usepackage{csc}
\title{Гарантии безопасности исключений}

\date{
   \textbf{CS центр}\\
   17 апреля 2018 \\
   Санкт-Петербург
}

\lstset{basicstyle=\fontsize{8pt}{0.85em}\ttfamily,  
        keywordstyle=\fontsize{8pt}{0.85em}\color{MOOCBlue}\bfseries, 
        commentstyle=\fontsize{8pt}{0.85em}\ttfamily\color{MOOCGreen}}

\begin{document}
\begin{frame} 
  \titlepage
\end{frame}
\newcommand{\fakeitem}{{\color{MOOCBlue}\textbullet}\ \ }



\begin{frame}[fragile]{Гарантии безопасности исключений}
    \begin{block}{Гарантия отсутствия исключений}
            “Ни при каких обстоятельствах функция  не будет генерировать исключения”.
    \end{block}\pause
    
    \begin{block}{Базовая гарантия}
            “При возникновении любого исключения 
            состояние программы останется
            согласованным”.
    \end{block}\pause

    \begin{block}{Строгая гарантия}
            “Если при выполнении операции возникнет исключение,  
            то программа останется том же в состоянии,  которое было до начала выполнения
            операции”.
    \end{block}
\end{frame}
\begin{frame}[fragile]{Строгая гарантия безопасности исключений}
    \begin{itemize}
        \pitem В каком случае мы не можем обеспечить строгую 
        гарантию безопасности исключений?
            
            \pause \medskip
            \emph{При наличии взаимодействия со внешним окружением.}
            \medskip\pause
        
        \item Как обеспечить строгую гарантию безопасности исключений  в остальных случаях?
            
            \pause \medskip
            \emph{Выполнять операцию над копией состояния программы.  Если операция прошла успешно, заменить состояние на копию.}
            \medskip\pause
            
        \item Когда можно обеспечить строгую гарантию {\em эффективно}?
        
            \pause \medskip
            \emph{Это вопрос архитектуры приложения.}
            
    \end{itemize}
\end{frame}
\begin{frame}[fragile]{Как добиться строгой гарантии?}
    \begin{lstlisting}
template<class T>
struct Array 
{
    void resize(size_t n) 
    {
        T * ndata = new T[n];
        for (size_t i = 0; i != n && i != size_; ++i)
            ndata[i] = data_[i];

        delete [] data_;
        data_ = ndata;
        size_ = n;    
    }

    T *     data_;
    size_t  size_;
};
    \end{lstlisting}
\end{frame}

\begin{frame}[fragile]{Как добиться строгой гарантии: вручную}
    \begin{lstlisting}
template<class T> 
struct Array 
{
    void resize(size_t n) 
    {
        T * ndata = new T[n];
        try { 
            for (size_t i = 0; i != n && i != size_; ++i)
                ndata[i] = data_[i];
        } catch (...) { 
            delete [] ndata;
            throw;
        }
        delete [] data_;
        data_ = ndata;
        size_ = n;    
    }
    
    T *     data_;
    size_t  size_;
};
    \end{lstlisting}
\end{frame}

\begin{frame}[fragile]{Как добиться строгой гарантии: RAII}
\begin{lstlisting}
template<class T>
struct Array 
{
    void resize(size_t n) 
    {
        unique_ptr<T[]> ndata(new T[n]);

        for (size_t i = 0; i != n && i != size_; ++i)
            ndata[i] = data_[i];

        data_ = std::move(ndata);
        size_ = n;    
    }

    unique_ptr<T[]> data_;
    size_t          size_;
};
\end{lstlisting}
\end{frame}

\begin{frame}[fragile]{Как добиться строгой гарантии: \texttt{swap}}
\begin{lstlisting}
template<class T>
struct Array 
{
    void resize(size_t n) 
    {
        Array t(n);
        for (size_t i = 0; i != n && i != size_; ++i)
            t[i] = data_[i];

        t.swap(*this);
    }

    T       * data_;
    size_t    size_;
};
\end{lstlisting}
\end{frame}


\begin{frame}[fragile]{Проектирование с учётом исключений}
    Рассмотрим традиционный интерфейс стека:
\begin{lstlisting}
template<class T> 
struct Stack 
{
    void push(T const& t) 
    { 
        data_.push_back(t); 
    }

    T pop() 
    {
        T tmp = data_.back();
        data_.pop_back();
        return tmp;
    }

    std::vector<T> data_;
};
\end{lstlisting}
\end{frame}

\begin{frame}[fragile]{Проектирование с учётом исключений}
    Рассмотрим традиционный интерфейс стека:
\begin{lstlisting}
template<class T> 
struct Stack 
{
    void push(T const& t) 
    { 
        data_.push_back(t); 
    }

    void pop(T & res) 
    {
        res = data_.back();
        data_.pop_back();
    }

    std::vector<T> data_;
};
\end{lstlisting}
\end{frame}

\begin{frame}[fragile]{Использование {\tt unique\_ptr}}
\begin{lstlisting}
template<class T> 
struct Stack 
{
    void push(T const& t) 
    { 
        data_.push_back(t); 
    }

    unique_ptr<T> pop() 
    {
        unique_ptr<T> tmp(new T(data_.back()));
        data_.pop_back();
        return tmp;
    }

    std::vector<T> data_;
};
\end{lstlisting}
\end{frame}


%\begin{frame}[fragile]{Проблемы с порядком операций}
%    Следите за порядком операций:
%\begin{lstlisting}
%void foo(unique_ptr<T> p, unique_ptr<V> q);
%\end{lstlisting}
%
%\begin{lstlisting}
%// потенциальный memory leak
%foo(unique_ptr<T>(new T()), unique_ptr<V>(new V()));
%
%// всё корректно
%unique_ptr<T> p(new T());
%foo(std::move(p), unique_ptr<V>(new V()));
%
%// корректно в C++14
%foo(make_unique<T>(), make_unique<V>());
%\end{lstlisting}
%\end{frame}


\begin{frame}[fragile]{Спецификация исключений}
Устаревшая возможность C++,
позволяющая указать список исключений, которые могут быть выброшены из функции.
\begin{lstlisting}
void foo() throw(std::logic_error) {
    if (...) throw std::logic_error();
    if (...) throw std::runtime_error();
}               
\end{lstlisting}
Если сработает второй {\tt if}, то программа аварийно завершится.
\begin{lstlisting}
void foo() {
    try {
        if (...) throw std::logic_error();
        if (...) throw std::runtime_error();
    } catch (std::logic_error & e) { 
        throw e;
    } catch (...) { 
        terminate();
    }
}
\end{lstlisting}
\end{frame}


\begin{frame}[fragile]{Ключевое слово \texttt{noexcept}}
\begin{itemize}
    \item Используется в двух значениях:
    \begin{itemize}
        \item Спецификатор функции, которая не бросает исключение.
        \item Оператор, проверяющий во время компиляции, что выражение
            специфицированно как небросающее исключение.
    \end{itemize}

\item Если функцию со спецификацией \code{noexcept} покинет  
    исключение, то стек не обязательно будет свёрнут,  перед тем как программа завершится.   
    В отличие от аналогичной ситуации с \code{throw}\texttt{()}.

\item Использование спецификации \code{noexcept} позволяет 
    компилятору лучше оптимизировать код, т.к. 
    не нужно  заботиться о сворачивании стека.
\end{itemize}

\end{frame}

\begin{frame}[fragile]{Использование \texttt{noexcept}}
    \begin{lstlisting}
void no_throw() noexcept;
void may_throw();

// копирующий конструктор noexcept
struct NoThrow { int m[100] = {}; };

// копирующий конструктор noexcept(false)
struct MayThrow { std::vector<int> v; };
    \end{lstlisting}

\begin{lstlisting}
MayThrow mt; 
NoThrow  nt;

bool a = noexcept(may_throw());  // false
bool b = noexcept(no_throw());   // true

bool c = noexcept(MayThrow(mt)); // false
bool d = noexcept(NoThrow(nt));  // true
\end{lstlisting}
\end{frame}

\begin{frame}[fragile]{Условный \texttt{noexcept}}
    В спецификации \texttt{noexcept} можно использовать 
    условные выражения времени компиляции.
\begin{lstlisting}
template <class T, size_t N>
void swap(T (&a)[N], T (&b)[N]) 
        noexcept(noexcept(swap(*a, *b)));

template <class T1, class T2>
struct pair {
    void swap(pair & p) 
        noexcept(noexcept(swap(first, p.first)) &&
                 noexcept(swap(second, p.second)))
    {
        swap(first, p.first);
        swap(second, p.second);
    }

    T1 first;
    T2 second;
};
\end{lstlisting}
\end{frame}

% TODO: добавить задачу про declval

\begin{frame}[fragile]{Зависимость от \texttt{noexcept}}
    Проверка \code{noexcept} используется в стандартной 
    библиотеке для обеспечения строгой гарантии безопасности исключений с помощью
    \texttt{std::move\_if\_noexcept} (например, \texttt{vector::push\_back}).

\begin{lstlisting}
struct Bad {
    Bad() {}
    Bad(Bad&&);      // может бросить
    Bad(const Bad&); // не важно
};
struct Good {
    Good() {}
    Good(Good&&) noexcept; // не бросает
    Good(const Good&) ;    // не важно
};
\end{lstlisting}
\begin{lstlisting}
Good g1;
Bad  b1;
Good g2 = std::move_if_noexcept(g1); // move
Bad  b2 = std::move_if_noexcept(b1); // copy
\end{lstlisting}
    
\end{frame}

\begin{frame}[fragile]{Заключение}
    \begin{itemize}
        \item Проектируйте архитектуру приложения с учётом ислючений.
        
        \pitem Функции, не бросающие исключения, нужно  объявлять
        как \code{noexcept}.
        
        \pitem Все использующие исключения функции  должны
        обеспечивать как минимум базовую  гарантию безопасности исключений.
        
        \pitem Там, где это возможно, старайтесь обеспечить 
        строгую гарантию безопасности исключений.
        
        \pitem Используйте \texttt{swap}, умные указатели и другие
        RAII объекты  для обеспечения строгой безопасности исключений.
    \end{itemize}
\end{frame}

\end{document}
