\documentclass{beamer}
\usepackage{csc}
\title{Ещё о нововведениях \langcpp 11/\langcpp 14}

\date{
   \textbf{CS центр}\\
   27 февраля 2018 \\
   Санкт-Петербург
}

\lstset{basicstyle=\fontsize{8pt}{0.92em}\ttfamily,  
        keywordstyle=\fontsize{8pt}{0.92em}\color{MOOCBlue}\bfseries, 
        commentstyle=\fontsize{8pt}{0.92em}\ttfamily\color{MOOCGreen}}

\begin{document}
\begin{frame} 
  \titlepage
\end{frame}

% TODO: сделать текстовый степ
%\begin{frame}[fragile]{Наследование от списка типов}
%\begin{lstlisting}
%template <typename... BaseClasses> 
%struct ClassName : BaseClasses... {
%    ClassName (BaseClasses&&... base_classes) 
%        : BaseClasses(base_classes)... 
%    {}
%};
%\end{lstlisting}
%
%Можно использовать \code{sizeof...} для определения количества %параметров. 
%\begin{lstlisting}
%template<typename ...Args> struct SomeStruct {
%    static const int count = sizeof...(Args);
%};
%\end{lstlisting}
%\end{frame}


\begin{frame}[fragile]{Ключевые слова \code{default} и \code{delete}}
\begin{lstlisting}
struct SomeType {
    SomeType() = default; // Конструктор по умолчанию.
    SomeType(OtherType value);
};

struct NonCopyable {
    NonCopyable() = default;
    NonCopyable(const NonCopyable&) = delete;
    NonCopyable & operator=(const NonCopyable&) = delete;
};
\end{lstlisting}
\pause\medskip
Удалять можно и обычные функции.
\begin{lstlisting}
template<class T>
void foo(T const * p) { ... }

void foo(char const *) = delete;
\end{lstlisting}
\end{frame}

\begin{frame}[fragile]{Делегация конструкторов}
    \begin{lstlisting}
struct SomeType  {
    SomeType(int newNumber): number(newNumber) {}
    SomeType() : SomeType(42) {}
private:
    int number;
};
struct SomeClass {
    SomeClass() {}
    explicit SomeClass(int newValue): value(newValue) {}
private:
    int value = 5;
};
struct BaseClass {
    BaseClass(int value);
};
struct DerivedClass : public BaseClass {
    using BaseClass::BaseClass;
};
\end{lstlisting}
\end{frame}


\begin{frame}[fragile]{Явное переопределение и финальность}
    \begin{lstlisting}
struct Base {
    virtual void update();
    virtual void foo(int);
    virtual void bar() const;
};
struct Derived : Base {
    void updata() override;          	  // error
    void foo(int) override;               // OK
    virtual void foo(long) override;      // error
    virtual void foo(int) const override; // error
    virtual int  foo(int) override;       // error
    virtual void bar(long);               // OK
    virtual void bar() const final;       // OK
};
struct Derived2 final : Derived {
    virtual void bar() const;       // error
};
struct Derived3 : Derived2 {};      // error
    \end{lstlisting}
\end{frame}

% TODO: слайд про enum-ы
%    \item Перечисления со строгой типизацией: 
%\begin{lstlisting}
%enum class Enum1 { Val1, Val2, Val3 = 100, Val4 };
%enum class Enum2 : unsigned int { Val1, Val2i };
%\end{lstlisting}


\begin{frame}[fragile]{Константные выражения}
Для констант и функций времени компиляции.
\begin{lstlisting}
constexpr double accOfGravity = 9.8;
constexpr double moonGravity = accOfGravity / 6;

constexpr int pow(int x, int k) 
{ return k == 0 ? 1 : x * pow(x, k - 1); }

int data[pow(3, 5)] = {};
\end{lstlisting}

\begin{lstlisting}
struct Point {
   double x, y;
   constexpr Point(double x = 0, double y = 0)
       : x(x), y(y) {}
   constexpr double getX() const { return x; }
   constexpr double getY() const { return y; }
};
constexpr Point p(moonGravity, accOfGravity);
constexpr auto x = p.getX();
\end{lstlisting}
\end{frame}

\begin{frame}[fragile]{Range-based {\tt for}}
Специально для работы с последовательностями.
\begin{lstlisting}
int array[] = {1, 4, 9, 16, 25, 36, 49};

int sum = 0;
// по значению
for (int x : array) {
    sum += x;
}
// по ссылке
for (int & x : array) {
    x *= 2;
}
\end{lstlisting}

Применим к встроенным массивам, спискам инициализации, контейнерам из стандартной библиотеки и любым другим типам, для которых определены функции \texttt{begin()} и \texttt{end()}, возвращающие итераторы  (об этом будет рассказано дальше).      
\end{frame}

\begin{frame}[fragile]{Списки инициализации}
Возможность передать в функцию список значений.
\begin{lstlisting}
// в конструкторах массивов и других контейнеров
template<typename T>
struct Array {
    Array(std::initializer_list<T> list);
};

Array<int> primes = {2, 3, 5, 7, 11, 13, 17};
\end{lstlisting}

\begin{lstlisting}
// в обычных функциях
int sum(std::initializer_list<int> list) {
    int result = 0;
    for (int x : list) 
        result += x;
    return result;
}

int s = sum({1, 1, 2, 3, 5, 8, 13, 21});
\end{lstlisting}
\end{frame}

\begin{frame}[fragile]{Универсальная инициализация}
\begin{lstlisting}
struct CStyleStruct {
    int x;
    double y;
};
struct CPPStyleStruct {
    CPPStyleStruct(int x, double y): x(x), y(y) {}
    int x;
    double y;
};
\end{lstlisting}

\begin{lstlisting}
CStyleStruct 	s1 = {19, 72.0};// инициализация по-старому
CPPStyleStruct  s2(19, 83.0); 	// вызов конструктора по-старому
\end{lstlisting}

\begin{lstlisting}
CStyleStruct 	s1{19, 72.0}; // инициализация по-новому
CPPStyleStruct  s2{19, 83.0}; // вызов конструктора по-новому
\end{lstlisting}

\begin{lstlisting}
// тип не обязателен
CStyleStruct getValue() { return {6, 4.2}; } 
\end{lstlisting}
\end{frame}

\begin{frame}[fragile]{Новые строковые литералы}
\begin{lstlisting}
u8"I'm a UTF-8 string."         // char[]
u"This is a UTF-16 string."     // char\_16\_t[]
U"This is a UTF-32 string."     // char\_32\_t[]
L"This is a wide-char string."  // wchar\_t[]

u8"This is a Unicode Character: \u2018."
u"This is a bigger Unicode Character: \u2018."
U"This is a Unicode Character: \U00002018."

R"(The String Data \ Stuff " )"
R"delimiter(The String Data \ Stuff " )delimiter"

LR"(Raw wide string literal \t (without a tab))"
u8R"XXX(I'm a "raw UTF-8" string.)XXX"
uR"*(This is a "raw UTF-16" string.)*"
UR"(This is a "raw UTF-32" string.)"
\end{lstlisting}
\end{frame}

\begin{frame}
\begin{center}
\huge Семантика перемещения
\end{center}
\end{frame}

% % % % % % % % % % % % % % % % % %
\begin{frame}[fragile]{Излишнее копирование}
\begin{lstlisting}
struct String {
    String() = default;
    String(String const & s);
    String & operator=(String const & s);
    //...
private:
    char * data_ = nullptr;
    size_t size_ = 0;
};
\end{lstlisting}

\begin{lstlisting}
String getCurrentDateString() {
    String date;
    // date заполняется "21 октября 2015 года"
    return date;
}
\end{lstlisting}

\begin{lstlisting}
String date = getCurrentDateString();
\end{lstlisting}
\end{frame}

% % % % % % % % % % % % % % %
\begin{frame}[fragile]{Перемещающий конструктор и перемещающий~оператор присваивания}
\begin{lstlisting}
struct String 
{
    String (String && s) // \&\& - rvalue reference
        : data_(s.data_)
        , size_(s.size_) {
        s.data_ = nullptr;
        s.size_ = 0;
    }
    String & operator = (String && s) {
        delete [] data_;
        data_ = s.data_;
        size_ = s.size_;
        s.data_ = nullptr;
        s.size_ = 0;
        return *this;
    }
};
\end{lstlisting}
\end{frame}                                

% % % % % % % % % % % % % % % %
\begin{frame}[fragile]{Перемещающие методы при помощи \code{swap}}
\begin{lstlisting}
#include<utility>

struct String 
{
    void swap(String & s) { 
        std::swap(data_, s.data_);
        std::swap(size_, s.size_);
    }

    String (String && s) { 
        swap(s);
    }
    
    String & operator = (String && s) {
        swap(s);
        return *this;
    }
};
\end{lstlisting}
\end{frame}

% % % % % % % % % % % % % % % %
\begin{frame}[fragile]{Использование перемещения}
\begin{lstlisting}
struct String {
    String() = default;
    String(String const & s);  // lvalue-reference
    String & operator=(String const & s);
    String(String && s);       // rvalue-reference
    String & operator=(String && s);
private:
    char * data_ = nullptr;
    size_t size_ = 0;
};
\end{lstlisting}

\begin{lstlisting}
String getCurrentDateString() {
    String date;
    // date заполняется "21 октября 2015 года"
    return std::move(date);
}
\end{lstlisting}

\begin{lstlisting}
String date = getCurrentDateString();
\end{lstlisting}
\end{frame}

% % % % % % % % % % % % % % % %
\begin{frame}[fragile]{Перегрузка с lvalue/rvalue ссылками}
При перегрузке перемещающий метод вызывается для временных объектов
и для явно перемещённых с помощью \texttt{std::move}.
\begin{lstlisting}
String a(String("Hello"));  // перемещение

String b(a);                // копирование

String c(std::move(b));     // перемещение

a = b;                      // копирование

b = std::move(c);           // перемещение

c = String("world");        // перемещение
\end{lstlisting}

Это касается и обычных методов и функций, \\ которые принимают
lvalue/rvalue-ссылки.  
\end{frame}


\begin{frame}[fragile]{Перемещающие особые методы}
    \begin{itemize}
        \item Особые методы класса:
            \begin{itemize}
                \item конструктор по умолчанию,
                \item конструктор копирования,
                \item оператор присваивания,
                \item деструктор,
                \item перемещающий конструктор,
                \item перемещающий оператор присваивания.
             \end{itemize}
                
        \pitem Перемещающие методы генерируются только, если в классе отсутствуют пользовательские копирующие операции, перемещающие операции и деструктор.

        \pitem Генерация копирующих методов для классов с пользовательским конструктором признана устаревшей.
    \end{itemize}
\end{frame}

\begin{frame}[fragile]{Пример: {\tt unique\_ptr}}
    \begin{lstlisting}
#include <memory>
#include "units.hpp"

void foo(std::unique_ptr<Unit> p);

std::unique_ptr<Unit> bar();

int main() {
    std::unique_ptr<Unit> p1(new Elf()); // захват указателя

    // теперь p2 владеет указателем
    std::unique_ptr<Unit> p2(std::move(p1));
    
    p1 = std::move(p2);  // владение передаётся p1
    
    foo(std::move(p1));  // p1 передаётся в foo
    
    p2 = bar();  // std::move не нужен  
} 
    \end{lstlisting}
\end{frame}

\begin{frame}[fragile]{Perfect forwarding}
    \begin{lstlisting}
// для lvalue
template<typename T, typename Arg> 
unique_ptr<T> make_unique(Arg & arg) { 
    return unique_ptr<T>(new T(arg));
} 

// для rvalue
template<typename T, typename Arg> 
unique_ptr<T> make_unique(Arg && arg) { 
    return unique_ptr<T>(new T(std::move(arg)));
} 
    \end{lstlisting}

\texttt{std::forward} позволяет записать это одной функцией.
    \begin{lstlisting}
template<typename T, typename Arg> 
unique_ptr<T> make_unique(Arg&& arg) { 
    return unique_ptr<T>(
        new T(std::forward<Arg>(arg)));
} 
    \end{lstlisting}
\end{frame}

\begin{frame}[fragile]{Variadic templates $+$ perfect forwarding}
Можно применить \texttt{std::forward} для списка параметров.
\begin{lstlisting}
template<typename T, typename ...Args> 
std::unique_ptr<T> make_unique(Args&&... args) {
    return std::unique_ptr<T>(
                new T(std::forward<Args>(args)...));
}
\end{lstlisting}
Теперь \texttt{make\_unique} работает для произвольного числа аргументов.
\begin{lstlisting}
auto p = make_unique<Array<string>>(10, string("Hello"));
\end{lstlisting}
\end{frame}

\end{document}


