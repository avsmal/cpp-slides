\documentclass{beamer}
\usepackage{csc}
\title{Нововведения \langcpp 11/\langcpp 14}

\date{
   \textbf{CS центр}\\
   20 февраля 2018 \\
   Санкт-Петербург
}

\lstset{basicstyle=\fontsize{8pt}{0.9em}\ttfamily,  
        keywordstyle=\fontsize{8pt}{0.9em}\color{MOOCBlue}\bfseries, 
        commentstyle=\fontsize{8pt}{0.9em}\ttfamily\color{MOOCGreen}}

\begin{document}
\begin{frame} 
  \titlepage
\end{frame}


\begin{frame}[fragile]{Стандартизация \langcpp}
    \begin{itemize}
        \pause\item[1983] Появление \langcpp.
        \pause\item[1998] Первый стандарт ISO/IEC 14882:1998.
        \pause\item[2003] Стандарт ISO/IEC 14882:2003, исправляющий недостатки стандарта \langcpp[98].
        \pause\item[2011] Стандарт ISO/IEC 14882:2011.
        \pause\item[2014] Стандарт ISO/IEC 14882:2014, исправляющий недостатки стандарта \langcpp[11].
        \pause\item[2017] В конце года опубликован стандарт \langcpp[17].
\end{itemize}
\end{frame}

\begin{frame}[fragile]{Основные принципы разработки стандарта}
    \begin{itemize}
        \pause\item поддержка совместимости с предыдущими стандартами;
        \pause\item улучшение техники программирования;
        \pause\item улучшение \langcpp с точки зрения дизайна; 
        \pause\item увеличение типобезопасности для обеспечения безопасной альтернативы для существующих опасных подходов;
        \pause\item увеличение производительности;
        \pause\item «не платить за то, что не используешь»;
        \pause\item введение новых возможностей через стандартную\\ библиотеку, а~не через ядро языка;                
        \pause\item сделать \langcpp проще для изучения\\
        (сохраняя возможности, используемые программистами-экспертами).
    \end{itemize}
\end{frame}

\begin{frame}[fragile]{Мелкие улучшения}
\begin{enumerate}
    \pitem Исправлена проблема с угловыми скобками: {\tt T<U<int{>}>}.

    \pitem Определены понятия ``тривиальный класс'' и ``класс со стандартным размещением''.
    
    \pitem Ключевое слово \code{explicit} для оператора приведения типа.
\begin{lstlisting}
explicit operator bool () { ... }
\end{lstlisting}

    \pitem Шаблонный \code{typedef}
\begin{lstlisting}
template<class A, class B, int N> 
class SomeType;

template<typename B>
using TypedefName = SomeType<double, B, 5>;
\end{lstlisting}
\pause
\begin{lstlisting}
typedef void (*OtherType)(double); 
using OtherType = void (*)(double);
\end{lstlisting}
\end{enumerate}
\end{frame}


\begin{frame}[fragile]{Мелкие улучшения (продолжение)}
\begin{enumerate} \setcounter{enumi}{4}
    \pitem Добавлен тип \code{long long int}.

    \pitem Добавлена библиотека поддержки типов: по типу на этапе компиляции можно узнавать его свойства\\ (см. заголовочный файл \texttt{<type\_traits>}).

    \pitem Добавлены операторы \code{alignof} и \code{alignas}.
\begin{lstlisting}
alignas(float) unsigned char c[sizeof(float)];
\end{lstlisting}

    \pitem Добавлен \code{static\_assert}
\begin{lstlisting}
template <class T>
void run(T * data, size_t n) {
   static_assert(std::is_signed<T>::value, 
                 "T is not signed.");
}
\end{lstlisting}

%\item Внешние шаблоны.
%\begin{lstlisting}
%//instantiate here
%template class std::vector<MyClass>; 
%
%//don't instantiate here
%extern template class std::vector<MyClass>; 
%\end{lstlisting}
%    \item \code{sizeof} без содания экземпляра
%\begin{lstlisting}
%struct SomeType { OtherType member; };
% 
%sizeof(SomeType::member); //only C++11
%\end{lstlisting}
\end{enumerate}
\end{frame}

\begin{frame}[fragile]{\code{nullptr}}
    В язык добавлены тип \code{std::nullptr\_t} и литерал \code{nullptr}.

\begin{lstlisting}          
void foo(int a)     { ... }

void foo(int * p)   { ... }

void bar() 
{
    foo(0); // вызов foo(int a)
    foo((int *) 0); // \langcpp[98]
    foo(nullptr);   // \langcpp[11]
}
\end{lstlisting}
\medskip
Тип \code{std::nullptr\_t} имеет единственное значение \code{nullptr},
которое неявно приводится к нулевому указателю\\ на любой тип.
\end{frame}

\begin{frame}[fragile]{Вывод типов}
\begin{lstlisting}
Array<Unit *> units;

for(size_t i = 0; i != units.size(); ++i) {
    // Unit *
    auto u = units[i]; 	    
    
    // Array<Item> const \& 
    decltype(u->items()) items = u->items();
    ...
\end{lstlisting}
\pause
\begin{lstlisting}
    auto a = items[0];        // a - Item  
    decltype(items[0]) b = a; // b - Item const \& 
    
    decltype(a)    c = a;     // c - Item
    decltype((a))  d = a;     // d - Item \&

    decltype(b)    e = b;     // e - Item const \&
    decltype((b))  f = b;     // f - Item const \&
\end{lstlisting}
\end{frame}

\begin{frame}[fragile]{Альтернативный синтаксис для функций}
\begin{lstlisting}
// RETURN\_TYPE = ?
template <typename A, typename B> 
RETURN_TYPE Plus(A a, B b) { return a + b; }
\end{lstlisting}
\begin{lstlisting}
// некорректно, a и b определены позже
template <typename A, typename B> 
decltype(a + b) Plus(A a, B b) { return a + b; }
\end{lstlisting}
%\begin{lstlisting}
%template <typename A, typename B> 
%    decltype(std::declval<const A&>() + std::declval<const B&>())
%        Plus(const A &a, const B &b) {
%    return a + b;
%}
%
\begin{lstlisting}
// \langcpp[11]
template <typename A, typename B> 
auto Plus(A a, B b) -> decltype(a + b) { 
    return a + b; 
}
\end{lstlisting}
\begin{lstlisting}
// \langcpp[14]
template <typename A, typename B> 
auto Plus(A a, B b) {
    return a + b;
}
\end{lstlisting}
\end{frame}

\begin{frame}[fragile]{Шаблоны с переменным числом аргументов}
    \begin{lstlisting}
void printf(char const *s) {
    while (*s) {
        if (*s == '%' && *(++s) != '%')
            // обработать ошибку
        std::cout << *s++;
    }
}
template<typename T, typename... Args>
void printf(char const *s, T value, Args... args) {
    while (*s) {
        if (*s == '%' && *(++s) != '%') {
            std::cout << value;
            printf(++s, args...);
            return;
        }
        std::cout << *s++;
    }
    // обработать ошибку
}
\end{lstlisting}
\end{frame}


\begin{frame}[fragile]{Кортежи}
\begin{lstlisting}
std::tuple<std::string, int, int> getUnitInfo(int id) {
    if (id == 0) return std::make_tuple("Elf",	 60, 9);
    if (id == 1) return std::make_tuple("Dwarf", 80, 6);
    if (id == 2) return std::make_tuple("Orc", 	 90, 3);
    //...
}
int main() {
    auto ui0 = getUnitInfo(0);
    std::cout << "race: "<< std::get<0>(ui0) << ", "
              << "hp: "  << std::get<1>(ui0) << ", "
              << "iq: "  << std::get<2>(ui0) << "\n";
    
    std::string race1; int hp1; int iq1;
    std::tie(race1, hp1, iq1) = getUnitInfo(1);
    std::cout << "race: " << race1 << ", "
              << "hp:   " << hp1   << ", "
              << "iq:   " << iq1   << "\n";
}
\end{lstlisting}
\end{frame}


\begin{frame}[fragile]{{\tt std::function}}
Универсальный класс для хранения указателей на функции,\\
указателей на методы и функциональных объектов.
\begin{lstlisting}
int mult (int x, int y) { return x * y; }

struct IntDiv { 
    int operator()(int x, int y) const { 
        return x / y; 
    } 
};
\end{lstlisting}
\begin{lstlisting}
std::function<int (int, int)> op; 
if ( OP == '*' )
    op = &mult;
else if ( OP == '/')
    op = IntDiv();
int result = op(7,8);
\end{lstlisting}
                  
Позволяет работать и с указателями на методы.
\end{frame}

\begin{frame}[fragile]{Лямбда-выражения}
\begin{lstlisting}
std::function<int (int, int)> op = 
    [](int x, int y) { return x / y; } // IntDiv

// то же, но с указанием типа возвращаемого значения
op = [](int x, int y) -> int { return x / y; }

// С++14
op = [](auto x, auto y) { return x / y; } 
\end{lstlisting}

Можно захватывать \emph{локальные} переменные.
\begin{lstlisting}
// захват по ссылке
int total = 0;
auto addToTotal = [&total](int x) { total += x; };

// захват по значению
auto subTotal = [total](int & x) { x -= total ; };

// Можно захватывать this
auto callUpdate = [this](){ this->update(); };
\end{lstlisting}
\end{frame}

\begin{frame}[fragile]{Различные виды захвата}
Могут быть разные типы захвата, в т.ч. смешанные: 
\verb![], [x, &y], [&], [=], [&, x], [=, &z]!\\
\medskip
Определение переменных \verb![x = std::move(y)]! (\langcpp[14]).
\medskip

Не стоит использовать захват по умолчанию \verb![&]! или \verb![=]!.
\begin{lstlisting}
std::function<bool(int)> getFilter(Checker const& c) {
    auto d  = c.getModulo();
    // захватывает ссылку на локальную переменную
    return [&] (int i) { return i % d == 0; }
}
\end{lstlisting}

\begin{lstlisting}
struct Checker {
    std::function<bool(int)> getFilter() const {
        // захватывает this, а не d
        return [=] (int x) { return x % d == 0; }
    }
    int d;
};
\end{lstlisting}
\end{frame}

\end{document}


