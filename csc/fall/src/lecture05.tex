\documentclass{beamer}
\usepackage{csc}
\title{Лекция 5. Структуры}

\date{
   \textbf{CS центр}\\
   26 сентября 2017 \\
   Санкт-Петербург
}

\begin{document}
\begin{frame} 
  \titlepage
\end{frame}

\begin{frame}[fragile]{Зачем группировать данные?}
    Какая должна быть сигнатура у функции, которая вычисляет длину отрезка на
    плоскости?
\begin{lstlisting}
double length(double x1, double y1, 
              double x2, double y2);
\end{lstlisting}
    А сигнатура функции, проверяющей пересечение отрезков?
\begin{lstlisting}
bool intersects(double x11, double y11, 
                double x12, double y12,
                double x21, double y21, 
                double x22, double y22,
                double * xi, double * yi);
\end{lstlisting}
Координаты точек являются логически связанными данными, которые всегда передаются
вместе.\\
Аналогично связанны координаты точек отрезка.
\end{frame}

\begin{frame}[fragile]{Структуры}
    Структуры~--- это способ синтаксически (и физически)
    сгруппировать логически связанные данные.
\begin{lstlisting}
struct Point {
    double x;
    double y;
};
struct Segment {
    Point p1;
    Point p2;
};

double length(Segment s);  

bool intersects(Segment s1, 
                Segment s2, Point * p);
\end{lstlisting}
\end{frame}

\begin{frame}[fragile]{Работа со структурами}
Доступ к полям структуры осуществляется через
оператор '\code{.}':
\begin{lstlisting}
#include <cmath>

double length(Segment s) {
    double dx = s.p1.x - s.p2.x;
    double dy = s.p1.y - s.p2.y;
    return sqrt(dx * dx + dy * dy);
}
\end{lstlisting}
Для указателей на структуры используется оператор '\code{->}'.
\begin{lstlisting}
double length(Segment * s) {
    double dx = s->p1.x - s->p2.x;
    double dy = s->p1.y - s->p2.y;
    return sqrt(dx * dx + dy * dy);
}
\end{lstlisting}
\end{frame}

\begin{frame}[fragile]{Инициализация структур}
Поля структур можно инициализировать подобно массивам:
\begin{lstlisting}
    Point p1  = { 0.4, 1.4 };
    Point p2  = { 1.2, 6.3 };
    Segment s = { p1,  p2  };
\end{lstlisting}
Структуры могут хранить переменные разных типов.
\begin{lstlisting}
struct IntArray2D {
    size_t a;
    size_t b;
    int ** data;
};
\end{lstlisting}
\begin{lstlisting}
    IntArray2D a = {n, m, create_array2d(n, m)};
\end{lstlisting}
\end{frame}

\begin{frame}[fragile]{Методы}
    Метод — это функция, определённая внутри структуры.
\small
\begin{lstlisting}
struct Segment {
    Point p1;
    Point p2;
    double length() {
        double dx = p1.x - p2.x;
        double dy = p1.y - p2.y;
        return sqrt(dx * dx + dy * dy);
    }
};
int main() {
    Segment s = { { 0.4, 1.4 }, { 1.2, 6.3 } };
    cout << s.length() << endl;
    return 0;
}
\end{lstlisting}
\end{frame}                            

\begin{frame}[fragile]{Методы}
    Методы реализованы как функции с неявным параметром \code{this},
    который указывает на текущий экземпляр структуры.
\small
\begin{lstlisting}
struct Point 
{
    double x;
    double y;

    void shift(/* Point * this, */
               double x, double y) {
        this->x += x;
        this->y += y;        
    }
};
\end{lstlisting}
\end{frame}                            

\begin{frame}[fragile]{Методы: объявление и определение}
    Методы можно разделять на объявление и определение:
\begin{lstlisting}
struct Point 
{
    double x;
    double y;

    void shift(double x, double y);
};
\end{lstlisting}

\begin{lstlisting}
void Point::shift(double x, double y) 
{
    this->x += x;
    this->y += y;        
}
\end{lstlisting}
\end{frame}                            

\begin{frame}[fragile]{Абстракция и инкапсуляция}
    Использование методов позволяет объединить данные и функции для работы с ними.
\small
\begin{lstlisting}
struct IntArray2D {
    int & get(size_t i, size_t j) {
        return data[i * b + j];        
    }
    size_t a;
    size_t b;
    int * data;
};
\end{lstlisting}

\begin{lstlisting}
    IntArray2D m = foo();
    for (size_t i = 0; i != m.a; ++i )
        for (size_t j = 0; j != m.b; ++j)
            if (m.get(i, j) < 0) m.get(i,j) = 0;
\end{lstlisting}
\end{frame}

\begin{frame}[fragile]{Конструкторы}
    Конструкторы — это методы для инициализации структур.
\begin{lstlisting}
struct Point {
    Point() { 
        x = y = 0;
    }
    Point(double x, double y) {
        this->x = x;
        this->y = y;
    }
    double x;
    double y;
};
\end{lstlisting}
\begin{lstlisting}
    Point p1;
    Point p2(3,7);
\end{lstlisting}
\end{frame}

\begin{frame}[fragile]{Список инициализации}
    Список инициализации позволяет проинициализировать поля до входа в
    конструктор.
\begin{lstlisting}
struct Point {
    Point() : x(0), y(0) 
    {}
    Point(double x, double y) : x(x), y(y) 
    {}

    double x;
    double y;
};
\end{lstlisting}
Инициализации полей в списке инициализации
происходит в {\em порядке объявления полей} в структуре.
\end{frame}

\begin{frame}[fragile]{Значения по умолчанию}
    \begin{itemize}
        \item Функции могут иметь значения параметров {\em по умолчанию}.
        \item Значения параметров по умолчанию нужно указывать в {\em объявлении
            функции}.
\begin{lstlisting}
struct Point {
    Point(double x = 0, double y = 0) 
        : x(x), y(y)
    {}
    double x;
    double y;
};
\end{lstlisting}
\begin{lstlisting}
    Point p1;
    Point p2(2);
    Point p3(3,4);
\end{lstlisting}
    \end{itemize}
\end{frame}

\begin{frame}[fragile]{Конструкторы от одного параметра}
    Конструкторы от одного параметра задают {\em неявное}
    пользовательское преобразование:
\begin{lstlisting}
struct Segment {
    Segment() {}
    Segment(double length) 
        : p2(length, 0)
    {}
    Point p1;
    Point p2;
};
\end{lstlisting}
\begin{lstlisting}
    Segment s1;
    Segment s2(10);
    Segment s3 = 20;
\end{lstlisting}
\end{frame}

\begin{frame}[fragile]{Конструкторы от одного параметра}
    Для того, чтобы запретить {\em неявное} пользовательское
преобразование, используется ключевое слово \code{explicit}.
\begin{lstlisting}
struct Segment {
    Segment() {}
    explicit Segment(double length) 
        : p2(length, 0)
    {}
    Point p1;
    Point p2;
};
\end{lstlisting}
\begin{lstlisting}
    Segment s1;
    Segment s2(10);
    Segment s3 = 20; // error
\end{lstlisting}
\end{frame}

\begin{frame}[fragile]{Конструкторы от одного параметра}
Неявное пользовательское преобразование, 
задаётся также конструкторами, которые могут принимать один параметр.
\begin{lstlisting}
struct Point {
    explicit Point(double x = 0, double y = 0) 
        : x(x), y(y)
    {}
    double x;
    double y;
};
\end{lstlisting}
\begin{lstlisting}
    Point p1;
    Point p2(2);
    Point p3(3,4);
    Point p4 = 5; // error
\end{lstlisting}
\end{frame}

\begin{frame}[fragile]{Конструктор по умолчанию}
    Если у структуры нет конструкторов, то конструктор 
    без параметров, {\em конструктор по умолчанию},
    генерируется компилятором.
\begin{lstlisting}
struct Segment {
    Segment(Point p1, Point p2) 
        : p1(p1), p2(p2)
    {}
    Point p1;
    Point p2;
};
\end{lstlisting}
\begin{lstlisting}
    Segment s1; // error
    Segment s2(Point(), Point(2,1));
\end{lstlisting}
\end{frame}

\begin{frame}[fragile]{Особенности синтаксиса \langcpp}

    {\em ``Если что-то похоже на объявление функции, то это
    и есть объявление функции.''}
\begin{lstlisting}
struct Point {
    explicit Point(double x = 0, double y = 0) 
        : x(x), y(y) {}
    double x;
    double y;
};
\end{lstlisting}
\begin{lstlisting}
    Point p1;   // определение переменной
    Point p2(); // объявление функции

    double k = 5.1;
    Point p3(int(k)); // объявление функции
    Point p4((int)k); // определение переменной
\end{lstlisting}
\end{frame}

\begin{frame}[fragile]{Деструктор}
    Деструктор — это метод, который вызывается при удалении структуры,
    генерируется компилятором.
\begin{lstlisting}
struct IntArray {
    explicit IntArray(size_t size)
        : size(size)
        , data(new int[size])
    { }

    ~IntArray() {
        delete [] data;
    }
    
    size_t size;
    int *  data;
};
\end{lstlisting}
\end{frame}

\begin{frame}[fragile]{Время жизни}{}
    {\em Время жизни}~--- это временной интервал 
    между вызовами конструктора и деструктора.
\begin{lstlisting}
void foo()
{
    IntArray a1(10); // создание a1
    IntArray a2(20); // создание a2 
    for (size_t i = 0; i != a1.size; ++i) {
        IntArray a3(30); // создание a3
        ...
    } // удаление a3
} // удаление a2, потом a1
\end{lstlisting}
Деструкторы переменных на стеке вызываются в обратном
порядке (по отношению к порядку вызова конструкторов).
\end{frame}

\begin{frame}[fragile]{Объекты и классы}
    \small
    \begin{itemize}
        \item Структуру с методами, конструкторами и деструктором
            называют {\em классом}.
        \item Экземпляр (значение) класса называется {\em объектом}.
\begin{lstlisting}
struct IntArray {
    explicit IntArray(size_t size);
    ~IntArray(); 
    int & get(size_t i);

    size_t size;
    int *  data;
};
\end{lstlisting}
\begin{lstlisting}
    IntArray a(10);
    IntArray b = {20, new int[20]}; // ошибка
\end{lstlisting}
    \end{itemize}
\end{frame}

\begin{frame}[fragile]{Объекты в динамической памяти}{Создание}
Для создания объекта в динамической памяти используется оператор \code{new},
он отвечает за вызов конструктора.
\begin{lstlisting}
struct IntArray {
    explicit IntArray(size_t size);

    size_t size_;
    int *  data_;
};
\end{lstlisting}
\begin{lstlisting}
    // выделение памяти и создание объекта
    IntArray * pa = new IntArray(10);
    // только выделение памяти
    IntArray * pb = 
        (IntArray *)malloc(sizeof(IntArray));
\end{lstlisting}
\end{frame}

\begin{frame}[fragile]{Объекты в динамической памяти}{Удаление}
    При вызове оператора \code{delete} вызывается
    деструктор объекта.
\begin{lstlisting}
    // выделение памяти и создание объекта
    IntArray * pa = new IntArray(10);

    // вызов деструктора и освобождение памяти
    delete pa;
\end{lstlisting}
    Операторы \code{new []} и \code{delete []} работают аналогично
\begin{lstlisting}
    // выделение памяти и создание 10 объектов
    // (вызывается конструктор по умолчанию)
    IntArray * pa = new IntArray[10];

    // вызов деструкторов и освобождение памяти
    delete [] pa;
\end{lstlisting}
\end{frame}

\begin{frame}[fragile]{Placement new}{}
\begin{lstlisting}
// выделение памяти
void * p = myalloc(sizeof(IntArray));

// создание объекта по адресу p
IntArray * a = new (p) IntArray(10);

// явный вызов деструктора 
a->~IntArray();

// освобождение памяти
myfree(p);
\end{lstlisting}
Проблемы с выравниванием:
\begin{lstlisting}
char b[sizeof(IntArray)];
new (b) IntArray(20); // потенциальная проблема
\end{lstlisting}
\end{frame}
\end{document}

\begin{frame}[fragile]{Модификаторы доступа}{}
    Модификаторы доступа позволяют ограничивать 
    доступ к методам и полям класса.
\begin{lstlisting}
struct IntArray {
    explicit IntArray(size_t size) 
        : size_(size), data_(new int[size]) 
    {}
    ~IntArray() { delete [] data_;  }

    int &  get(size_t i) { return data_[i]; }
    size_t size()        { return size_; }

private:
    size_t size_;
    int *  data_;
};
\end{lstlisting}
\end{frame}

\begin{frame}[fragile]{Ключевое слово \code{class}}{}
    Ключевое слово \code{struct} можно заменить на 
    \code{class}, тогда поля и методы
    по умолчанию будут private.

\begin{lstlisting}
class IntArray {
public:
    explicit IntArray(size_t size) 
        : size_(size), data_(new int[size]) 
    {}
    ~IntArray() { delete [] data_;  }

    int &  get(size_t i) { return data_[i]; }
    size_t size()        { return size_; }

private:
    size_t size_;
    int *  data_;
};
\end{lstlisting}
\end{frame}

\begin{frame}[fragile]{Инварианты класса}{}
    \begin{itemize}
        \item Выделение {\em публичного интерфейса} позволяет
            поддерживать {\em инварианты класса}\\
            (сохранять
            данные объекта в согласованном состоянии).
\begin{lstlisting}
struct IntArray {
    ...
    size_t size_;
    int *  data_; // массив размера size\_
};
\end{lstlisting}
        
        \item Для сохранения инвариантов класса:
            \begin{enumerate}
                \item все поля должны быть закрытыми, 
                \item публичные методы должны сохранять инварианты класса.
            \end{enumerate}

        \item Закрытие полей класса позволяет абстрагироваться 
            от способа хранения данных объекта.
    \end{itemize}
\end{frame}

\begin{frame}[fragile]{Публичный интерфейс}{}
\begin{lstlisting}
struct IntArray {
    ...
    void resize(size_t nsize) {
        int * ndata = new int[nsize];
        size_t n = nsize > size_ ? size_ : nsize;
        for (size_t i = 0; i != n; ++i)
            ndata[i] = data_[i];
        delete [] data_;
        data_ = ndata;
        size_ = nsize;
    }
private:
    size_t size_;
    int *  data_;
};
\end{lstlisting}
\end{frame}

\begin{frame}[fragile]{Абстракция}{}
\begin{lstlisting}
struct IntArray {
public:
    explicit IntArray(size_t size) 
        : size_(size), data_(new int[size]) 
    {}
    ~IntArray() { delete [] data_;  }

    int &  get(size_t i) { return data_[i]; }
    size_t size()        { return size_; }

private:
    size_t size_;
    int *  data_;
};
\end{lstlisting}
\end{frame}

\begin{frame}[fragile]{Абстракция}{}
\begin{lstlisting}
struct IntArray {
public:
    explicit IntArray(size_t size) 
        : data_(new int[size + 1]) 
    {
        data_[0] = size;
    }
    ~IntArray() { delete [] data_;  }

    int &  get(size_t i) { return data_[i + 1]; }
    size_t size()        { return data_[0]; }

private:
    int *  data_;
};
\end{lstlisting}
\end{frame}

\end{document}
