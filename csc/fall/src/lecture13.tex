\documentclass{beamer}
\usepackage{csc}
\title{Лекция 13. Шаблоны, часть вторая}

\date{
   \textbf{CS центр}\\
   5 декабря 2017 \\
   Санкт-Петербург
}

\lstset{basicstyle=\fontsize{7pt}{1em}\ttfamily,  
        keywordstyle=\fontsize{7pt}{1em}\color{MOOCBlue}\bfseries, 
        commentstyle=\fontsize{7pt}{1em}\ttfamily\color{MOOCGreen}}

\begin{document}
\begin{frame} 
  \titlepage
\end{frame}

\begin{frame}[fragile]{Полная специализация шаблонов: классы}
\small
    \begin{lstlisting}
template<class T>
struct Array {
    ...
    T *   data_;
};
template<>
struct Array<bool> {
    static unsigned const BITS = 8 * sizeof(unsigned);
    explicit Array(size_t size)
        : size_(size)
        , data_(new unsigned[size_ / BITS + 1])
    {}
    bool operator[](size_t i) const {
        return data_[i / BITS] & (1 << (i % BITS)); 
    }
private:
    size_t  size_;
    unsigned *  data_;
};
    \end{lstlisting}
\end{frame}

\begin{frame}[fragile]{Полная специализация шаблонов: функции}
\small
    \begin{lstlisting}
template<class T>
void swap(T & a, T & b) 
{
    T tmp(a);
    a = b;
    b = tmp;
}

template<>
void swap<Database>(Database & a, Database & b) 
{
    a.swap(b);
}

template<class T>
void swap(Array<T> & a, Array<T> & b) 
{
    a.swap(b);
}
    \end{lstlisting}
\end{frame}

\begin{frame}[fragile]{Специализация шаблонов и перегрузка}
\small
    \begin{lstlisting}
template<class T>
void foo(T a, T b) { cout << "same types" << endl; }

template<class T, class V>
void foo(T a, V b) { cout << "different types" << endl; }

template<>
void foo<int, int>(int a, int b) { 
    cout << "both parameters are int" << endl; 
} 

int main() {
    foo(3, 4);
    return 0;    
}
    \end{lstlisting}
\end{frame}

\begin{frame}[fragile]{Частичная специализация шаблонов}
\small
    \begin{lstlisting}
template<class T>
struct Array {
    T & operator[](size_t i) { return data_[i]; }
    ...
};

template<class T>
struct Array<T *> {
    explicit Array(size_t size)
        : size_(size)
        , data_(new T *[size_])
    {}
    
    T & operator[](size_t i) { return *data_[i]; }

private:
    size_t  size_;
    T **    data_;
};
    \end{lstlisting}
\end{frame}

\begin{frame}[fragile]{Нетиповые шаблонные параметры}
\small
Параметрами шаблона могут быть типы, целочисленные значения, 
указатели/ссылки на значения с внешней линковкой и шаблоны.
    \begin{lstlisting}
template<class T, size_t N, size_t M>
struct Matrix {
    ...
    T & operator()(size_t i, size_t j) 
    { return data_[M * j + i]; }
private:
    T data_[N * M];
};

template<class T, size_t N, size_t M, size_t K>
Matrix<T, N, K> operator*(Matrix<T, N, M> const& a, 
                          Matrix<T, M, K> const& b);

// log - это глобальная переменная
template<ofstream & log>
struct FileLogger { ... };
    \end{lstlisting}
\end{frame}

\begin{frame}[fragile]{Шаблонные параметры~--- шаблоны}
\small
    \begin{lstlisting}
// int --> string
string toString( int i );

// работает только с Array<>
Array<string> toStrings( Array<int> const& ar ) {
    Array<string> result(ar.size());
    for (size_t i = 0; i != ar.size(); ++i)
        result.get(i) = toString(ar.get(i));
    return result;
}

// от контейнера требуются: конструктор от size, методы size() и get()
template<template <class> class Container>
Container<string> toStrings(Container<int> const& c) {
    Container<string> result(c.size());
    for (size_t i = 0; i != c.size(); ++i)
        result.get(i) = toString(c.get(i));
    return result;
}
    \end{lstlisting}
\end{frame}

\begin{frame}[fragile]{Использование зависимых имён}
\small
    \begin{lstlisting}
template<class T>
struct Array {
    typedef T value_type;
    ...
private:
    size_t  size_;
    T *     data_;
};

template<class Container>
bool contains(Container const& c, 
              typename Container::value_type const& v);

int main()
{
    Array<int> a(10);
    contains(a, 5);
    return 0;
}

    \end{lstlisting}
\end{frame}

\begin{frame}[fragile]{Компиляция шаблонов}
\small
    \begin{itemize}
        \item Шаблон независимо компилируется для каждого значения шаблонных
            параметров.

        \item Компиляция ({\em инстанциирование}) шаблона происходит в точке первого использования~---
            {\em точке инстанциирования шаблона}.

        \item Компиляция шаблонов классов~— ленивая, компилируются только те методы,
            которые используются.
        
        \item В точке инстанциирования шаблон должен быть полностью определён.

        \item Шаблоны следует определять в заголовочных файлах.

        \item Все шаблонные функции (свободные функции и методы) являются
            \code{inline}.

        \item В разных единицах трансляции инстанциирование происходит
            независимо.
    \end{itemize}
\end{frame}

\begin{frame}[fragile]{Резюме про шаблоны}
\small
    \begin{itemize}
        \item
            Большие шаблонные классы следует разделять на два заголовочных файла:
            объявление (\texttt{array.hpp}) и определение (\texttt{array\_impl.hpp}).

        \item Частичная специализация и шаблонные параметры по умолчанию
            есть только у шаблонов классов.

        \item Вывод шаблонных параметров есть только у шаблонов функций.

        \item Предпочтительно использовать перегрузку шаблонных функций вместо
            их полной специализации.

        \item Полная специализация функций~--- это обычные функции.
    
        \item Виртуальные методы, конструктор по умолчанию, конструктор
            копирования, оператор присваивания и деструктор не могут быть
            шаблонными.
        \item Используйте \code{typedef} для длинных шаблонных имён.
    \end{itemize}
\end{frame}
\end{document}

 
