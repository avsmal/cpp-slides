\documentclass{beamer}
\usepackage{csc}
\title{Лекция 3. Массивы, указатели и ссылки}

\date{
   \textbf{CS центр}\\
   12 сентября 2017 \\
   Санкт-Петербург
}

\begin{document}
\begin{frame} 
  \titlepage
\end{frame}

\begin{frame}[fragile]{Указатели}
    \begin{itemize}
        \item Указатель — это переменная, хранящая адрес некоторой 
            ячейки памяти.
           
        \item Указатели являются типизированными.
\begin{lstlisting}
int   i = 3; // переменная типа int
int * p = 0; // указатель на переменную типа int
\end{lstlisting}
        
        \item Нулевому указателю (которому присвоено значение \code{0}) не~соответствует никакая ячейка памяти.

        \item Оператор взятия адреса переменной \code{\&}.

        \item Оператор разыменования \code{*}.

\begin{lstlisting}
p  = &i; // указатель p указывает на переменную i
*p = 10; // изменяется ячейка по адресу p, т.е. i
\end{lstlisting}
    \end{itemize}

\end{frame}

\begin{frame}[fragile]{Передача параметров по указателю}
Рассмотрим функцию, меняющую параметры местами:
\begin{lstlisting}
void swap (int a, int b) {
    int t = a;
    a = b;
    b = t;
}

int main() {
    int k = 10, m = 20;
    swap (k, m);
    cout << k << ' ' << m << endl; // 10 20
    return 0;
}
\end{lstlisting}

{\tt swap} изменяет локальные копии переменных {\tt k} и {\tt m}.
\end{frame}

\begin{frame}[fragile]{Передача параметров по указателю}
    Вместо значений типа \code{int} будем передавать указатели.
\begin{lstlisting}
void swap (int * a, int * b) {
    int t = *a;
    *a = *b;
    *b = t;
}

int main() {
    int k = 10, m = 20;
    swap (&k, &m);
    cout << k << ' ' << m << endl; // 20 10
    return 0;
}
\end{lstlisting}

{\tt swap} изменяет переменные {\tt k} и {\tt m} по указателям на них.
\end{frame}


\begin{frame}[fragile]{Массивы}
    \begin{itemize}
        \item Массив~— это набор однотипных элементов, расположенных в памяти
            друг за другом, доступ к которым осуществляется по индексу.

        \item \langcpp позволяет определять массивы на стеке.
\begin{lstlisting}
// массив 1 2 3 4 5 0 0 0 0 0
int m[10] = {1, 2, 3, 4, 5}; 
\end{lstlisting}

        \item Индексация массива начинается с \code{0}, последний элемент
            массива длины {\tt n} имеет индекс {\tt n - 1}.
\begin{lstlisting}
for (int i = 0; i < 10; ++i)
    cout << m[i] << ' '; 
cout << endl;
\end{lstlisting}
    \end{itemize}
\end{frame}

\begin{frame}[fragile]{Связь массивов и указателей}
    \begin{itemize}
        \item Указатели позволяют передвигаться по массивам.
        \item Для этого используется арифметика указателей:
\begin{lstlisting}
int m[10] = {1, 2, 3, 4, 5}; 
int * p = &m[0]; // адрес начала массива
int * q = &m[9]; // адрес последнего элемента
\end{lstlisting}
            \begin{itemize}
                \item {\tt (p + k)} — сдвиг на {\tt k} ячеек типа \code{int} вправо.
                \item {\tt (p - k)} — сдвиг на {\tt k} ячеек типа \code{int} влево.
                \item {\tt (q - p)} — количество ячеек между
                    указателями.
                \item {\tt p[k]} эквивалентно {\tt *(p + k)}.
            \end{itemize}
    \end{itemize}
\begin{center}
\begin{tikzpicture}[
      start chain=1 going right,node distance=-0.15mm
    ]
    \tikzstyle{cell} = [rectangle,minimum width=5mm, minimum height=5mm,draw,on chain=1];
    \tikzstyle{arrow} = [>=stealth',shorten >=2pt, shorten <=2pt]

    \foreach \i in {1,2}
        \node[cell,dashed] {};
        
    \node[cell,dashed] (b) {};

    \foreach \i in {1,2,3,4,5}
        \node[cell,thick] (m\i) {\i};
        
    \foreach \i in {0,0,0}
        \node[cell,thick] {\i};

    \node[cell,thick] (e) {0};
    \node[cell,thick] {0};

    \node[below of=b,yshift=-10mm] (p) {{\tt p}};
    \path [arrow,->] (p) edge node {} (b.south east);

    \node[below of=e,yshift=-10mm] (q) {{\tt q}};
    \path [arrow,->] (q) edge node {} (e.south east);

    \node[below of=m2,yshift=-10mm] (p2) {{\tt (p + 2)}};
    \path [arrow,->] (p2) edge node {} (m2.south east);

    \node[below of=m5,yshift=-10mm] (q4) {{\tt (q - 4)}};
    \path [arrow,->] (q4) edge node {} (m5.south east);

    \foreach \i in {1,2,3}
        \node[cell,dashed] {};
\end{tikzpicture}
\end{center}
\end{frame}

\begin{frame}[fragile]{Примеры}
Заполнение массива:
\begin{lstlisting}
int m[10] = {}; // изначально заполнен нулями
// \hspace{2cm} \&m[0] \hspace{13mm}  \&m[9]
for (int * p = m ; p <= m + 9; ++p )
    *p = (p - m) + 1;    
// Массив заполнен числами от 1 до 10
\end{lstlisting}
Передача массива в функцию:
\begin{lstlisting} 
int max_element (int * m, int size) {
    int max = *m; 
    for (int i = 1; i < size; ++i)
        if (m[i] > max)
            max = m[i];
    return max;
}
\end{lstlisting}
\end{frame}


\begin{frame}[fragile]{Два способа передачи массива}
\small
\begin{lstlisting}
bool contains(int * m, int size, int value) {
    for (int i = 0; i != size; ++i)
        if (m[i] == value)
            return true;
    return false;
}
bool contains(int * p, int * q, int value) {
    for (; p != q; ++p)
        if (*p == value)
            return true;
    return false;
}
\end{lstlisting}
\begin{center}
\begin{tikzpicture}[
      start chain=1 going right,node distance=-0.15mm
    ]
    \tikzstyle{cell} = [rectangle,minimum width=5mm, minimum height=5mm,draw,on chain=1];
    \tikzstyle{arrow} = [>=stealth',shorten >=2pt, shorten <=2pt]

    \foreach \i in {1,2}
        \node[cell,dashed] {};
        
    \node[cell,dashed] (b) {};

    \foreach \i in {1,2,3,4,5}
        \node[cell,thick] (m\i) {\i};
        
    \foreach \i in {0,0,0}
        \node[cell,thick] {\i};

    \node[cell,thick] {0};
    \node[cell,thick] (e) {0};

    \node[below of=b,yshift=-10mm] (p) {{\tt p}};
    \path [arrow,->] (p) edge node {} (b.south east);

    \node[below of=e,yshift=-10mm] (q) {{\tt q}};
    \path [arrow,->] (q) edge node {} (e.south east);

    \node[cell,dashed,color=red] {};

    \foreach \i in {2,3}
        \node[cell,dashed] {};
\end{tikzpicture}
\end{center}
\end{frame}

\begin{frame}[fragile]{Возрат указателя из функции}
Функция для поиска максимума в массиве:
\begin{lstlisting}
int max_element (int * p, int * q) {
    int max = *p;
    for (; p != q; ++p)
        if (*p > max)
            max = *p;

    return max;
}
\end{lstlisting}
\begin{lstlisting}
    int m[10] = {...};
    int max = max_element(m, m + 10); 
    cout << "Maximum = " << max << endl; 
\end{lstlisting}
\end{frame}

\begin{frame}[fragile]{Возрат указателя из функции}
Функция для поиска максимума в массиве:
\begin{lstlisting}
int * max_element (int * p, int * q) {
    int * pmax = p;
    for (; p != q; ++p)
        if (*p > *pmax)
            pmax = p;

    return pmax;
}
\end{lstlisting}
\begin{lstlisting}
    int m[10] = {...};
    int * pmax = max_element(m, m + 10); 
    cout << "Maximum = " << *pmax << endl; 
\end{lstlisting}
\end{frame}

\begin{frame}[fragile]{Возрат значения через указатель}
Функция для поиска максимума в массиве:
\begin{lstlisting}
bool max_element (int * p, int * q, int * res) {
    if (p == q)
        return false;
    *res = *p;
    for (; p != q; ++p)
        if (*p > *res)
            *res = *p;
    return true;
}
\end{lstlisting}
\begin{lstlisting}
    int m[10] = {...};
    int max = 0;
    if (max_element(m, m + 10, &max)) 
        cout << "Maximum = " << max << endl; 
\end{lstlisting}
\end{frame}

\begin{frame}[fragile]{Возрат значения через указатель на указатель}
Функция для поиска максимума в массиве:
\begin{lstlisting}
bool max_element (int * p, int * q, int ** res) {
    if (p == q)
        return false;
    *res = p;
    for (; p != q; ++p)
        if (*p > **res)
            *res = p;
    return true;
}
\end{lstlisting}
\begin{lstlisting}
    int m[10] = {...};
    int * pmax = 0;
    if (max_element(m, m + 10, &pmax)) 
        cout << "Maximum = " << *pmax << endl; 
\end{lstlisting}
\end{frame}

\begin{frame}[fragile]{Недостатки указателей}
    \begin{itemize}
        \item Использование указателей синтаксически загрязняет код
            и усложняет его понимание.
            (Приходится использовать операторы \code{*} и \code{\&}.)

        \item Указатели могут быть неинициализированными (некорректный код).

        \item Указатель может быть нулевым (корректный код), 
            а значит указатель нужно проверять на равенство нулю.

        \item Арифметика указателей может сделать из корректного указателя
            некорректный (легко промахнуться).
    \end{itemize}
\end{frame}

\begin{frame}[fragile]{Ссылки}
    \begin{itemize}
        \item Для того, чтобы исправить некоторые недостатки указателей,
            в \langcpp введены ссылки.

        \item Ссылки являются ``красивой обёрткой'' над указателями:
\begin{lstlisting}
void swap (int & a, int & b) {
    int t = b;
    b = a;
    a = t;
}

int main() {
    int k = 10, m = 20;
    swap (k, m);
    cout << k << ' ' << m << endl; // 20 10
    return 0;
}
\end{lstlisting}
    \end{itemize}
\end{frame}

\begin{frame}[fragile]{Различия ссылок и указателей}
    \begin{itemize}
        \item Ссылка не может быть неинициализированной.
\begin{lstlisting}
int * p; // OK
int & l; // ошибка
\end{lstlisting}

        \item У ссылки нет нулевого значения.            
\begin{lstlisting}
int * p = 0; // OK
int & l = 0; // ошибка
\end{lstlisting}
            
    \item Ссылку нельзя переприсвоить:
\begin{lstlisting}
int a = 10, b = 20;
int * p = &a; // p указывает на a
p = &b;       // p указывает на b
int & l = a;  // l ссылается на a
l = b;        // a присваивается значение b
\end{lstlisting}
\end{itemize}
\end{frame}

\begin{frame}[fragile]{Различия ссылок и указателей}
    \begin{itemize}
        \item Нельзя получить адрес ссылки или ссылку на ссылку.
\begin{lstlisting}
int a = 10;
int * p = &a;  // p указывает на a
int ** pp = &p;// pp указывает на переменную p
int & l = a;   // l ссылается на a
int * pl = &l; // pl указывает на переменную a
int && ll = l; // ошибка
\end{lstlisting}

        \item Нельзя создавать массивы ссылок.
\begin{lstlisting}
int * mp[10] = {};  // массив указателей на int
int & ml[10] = {};  // ошибка
\end{lstlisting}

        \item Для ссылок нет арифметики.

    \end{itemize}
\end{frame}

\begin{frame}[fragile]{lvalue и rvalue}
    \begin{itemize}
        \item Выражения в \langcpp можно разделить на два типа:
            \begin{enumerate}
                \item {\bf lvalue}~--- выражения, значения которых являются
                    {\em ссылкой} на переменную/элемента массива, 
                    а значит могут быть указаны слева от оператора \code{=}.
                    
                \item {\bf rvalue}~--- выражения, значения которых являются
                    временными и не соответствуют никакой переменной/элементу
                    массива.
            \end{enumerate}

        \item Указатели и ссылки могут указывать только на lvalue.
\begin{lstlisting}
int a = 10, b = 20;
int m[10] = {1,2,3,4,5,5,4,3,2,1};
int & l1 = a;       // OK
int & l2 = a + b;   // ошибка
int & l3 = *(m + a / 2);        // OK
int & l4 = *(m + a / 2) + 1;    // ошибка
int & l5 = (a + b > 10) ? a : b;  // OK
\end{lstlisting}
\end{itemize}
\end{frame}

\begin{frame}[fragile]{Время жизни переменной}
    Следует следить за временем жизни переменных.
\begin{lstlisting}
int * foo() {
    int a = 10;
    return &a;
}

int & bar() {
    int b = 20;
    return b;
}
\end{lstlisting}
\begin{lstlisting}
    int * p = foo();
    int & l = bar();
\end{lstlisting}

\end{frame}

\end{document}
