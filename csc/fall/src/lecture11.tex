\documentclass{beamer}
\usepackage{csc}
\title{Лекция 11. Указатели на функции, методы и члены данных}

\date{
   \textbf{CS центр}\\
   24 октября 2017 \\
   Санкт-Петербург
}

\lstset{basicstyle=\fontsize{7pt}{1em}\ttfamily,  
        keywordstyle=\fontsize{7pt}{1em}\color{MOOCBlue}\bfseries, 
        commentstyle=\fontsize{7pt}{1em}\ttfamily\color{MOOCGreen}}

\begin{document}
\begin{frame} 
  \titlepage
\end{frame}

\begin{frame}[fragile]{Указатели на функции}
    Кроме указателей на значения в \texttt{C++} присутствуют три особенных 
    типа указателей:
    \begin{enumerate}
        \item указатели на функции (унаследованно из \texttt{C}),
        \item указатели на методы,
        \item указатели на поля классов.
    \end{enumerate}
    
    Указатели на функции (и методы) используются для
    \begin{enumerate}
        \item параметризация алгоритмов,
        \item обратных вызовов (callback),
        \item подписки на события (шаблон Listener),
        \item создание очередей событий/заданий.
    \end{enumerate}
\end{frame}

\begin{frame}[fragile]{Указатели на функции: параметризация алгоритмов}
\small
    \begin{lstlisting}
void qsort (void* base, size_t num, size_t size,
           int (*compar)(const void*,const void*));
    \end{lstlisting}

Как это работает?
    \begin{lstlisting}
bool less   (double a, double b) { return a < b; }
bool greater(double a, double b) { return a > b; }

void sort(double * p, double * q, bool (*cmp)(double, double)) {
    for (double * m = p; m != q; ++m)
        for (double * r = m; r + 1 != q; ++r)
            if ( cmp(*(r + 1), *r) )
                swap(*r, *(r + 1));
}
int main() {
    double m[100];
    sort(m, m + 100, &less);
    sort(m, m + 100, &greater);
}
    \end{lstlisting}
\end{frame}

\begin{frame}[fragile]{Указатели на функции: создание потока}
    \begin{lstlisting}
#include <pthread.h>
int pthread_create(pthread_t *thread, const pthread_attr_t *attr,
                   void *(*start_routine) (void *), void *arg);

double somedata[10][100];

void * thread_fun(void * data) {
    size_t const N = *((size_t *) data);
    sort(somedata[N], somedata[N] + 100);
    return 0;
}
int main( ) {
    pthread_t threads[10];
    size_t id[] = {0,1,2,3,4,5,6,7,8,9};
    for (size_t i = 0; i != 10; ++i)
        pthread_create(&threads[i], 0, &thread_fun, &id[i]);
    ...
}
    \end{lstlisting}
\end{frame}
 
\begin{frame}[fragile]{Сразу о полезности \texttt{typedef}}
Что здесь объявлено?
    \begin{lstlisting}
char * (*func(int, int))(int, int, int *, float);
    \end{lstlisting}
\pause

Функция двух целочисленных параметров, возвращающая указатель на функцию,
которая возвращает указатель на \texttt{char} и имеет собственный список
формальных параметров вида: \texttt{(int, int, int *, float)}

\medskip\pause

Как стоило это написать:
    \begin{lstlisting}
typedef char* (*SomeFunction)(int, int, int *, float);

SomeFunction func(int, int);
    \end{lstlisting}
\end{frame}

\begin{frame}[fragile]{Указатели на методы}
\small
В отличие от указателей на функции требуют экземпляр класса.
    \begin{lstlisting}
struct Person {
        string name()       const;
        string surname()    const;
        string address()    const;
};
typedef string (Person::*person_method)() const;

void print(Person const& p) {
    static person_method im[3] = 
        {&Person::name, &Person::surname, &Person::address};

    for (size_t i = 0; i != 3; ++i)
        cout << (p.*im[i])();
}                               // person\_method mem  
void sort(Person * p, Person * q, string (Person::*mem) () const);

sort(persons, persons + 100, &Person::name);   
sort(persons, persons + 100, &Person::surname);
    \end{lstlisting}
\end{frame}


\begin{frame}[fragile]{Указатели на члены данных}
\small
    Похожи на указатели на методы.
    \begin{lstlisting}
struct Person {
        string name;
        string surname;
        string address;
};
typedef string Person::*person_field;

void print(Person const& p) {
    static person_field im[3] = 
        {&Person::name, &Person::surname, &Person::address};

    for (size_t i = 0; i != 3; ++i)
        cout << (p.*im[i]);
}                              // person\_field mem
void sort(Person * p, Person * q, string Person::*mem);

sort(persons, persons + 100, &Person::name);   
sort(persons, persons + 100, &Person::surname);
    \end{lstlisting}
\end{frame}

\begin{frame}[fragile]{Резюме по синтаксису}
\small    
 Указатели на функции. 
    \begin{lstlisting}
int foo(double d) { return 0; }
int main() {
    int (*fptr)(double) = foo;
    return fptr(3.5);
}
    \end{lstlisting}

Указатели на методы и поля класса. 
    \begin{lstlisting}
struct Student { 
    string name () const { return name_; } 
    string name_;
};
int main() {
    string (Student::*mptr)() const = &Student::name;
    string  Student::*dptr          = &Student::name_;
    Student s;
    Student * p = &s;
    (s.*mptr)() == (p->*mptr)();
    (s.*dptr)   == (p->*dptr);
}
    \end{lstlisting}

\end{frame}

\begin{frame}[fragile]{Шаблон Listener}
    Решение с помощью ООП:
\begin{lstlisting}
struct Button;

struct ButtonListener {
    virtual void onButtonClick(Button * b, bool down) = 0;
    virtual ~ButtonListener(){}
};
struct Button {
    void subscribe( ButtonListener * bl );
};
    \end{lstlisting}

Решение с помощью указателей на функции:
    \begin{lstlisting}
struct Button;
typedef void (*ButtonProc)(Button *, bool, void *);
struct Button {
    void subscribe( ButtonProc bp, void * arg );
};
    \end{lstlisting}
\end{frame}

\begin{frame}[fragile]{Как такие указатели устроены?}
    \begin{block}{Что хранится в указателе на функцию?}
        \pause Хранится адрес функции.
    \end{block}

    \begin{block}{Что хранится в указателе на поле класса?}
        \pause
        Хранится смещение поля от начала объекта.
    \end{block}

    \begin{block}{Что хранится в указателе на метод?}
        \pause
        Там хранятся:
        \begin{enumerate}
            \item адрес метода,
            \item номер в таблице виртуальных методов,
            \item смещение.
        \end{enumerate}
    \end{block}

\end{frame}

\begin{frame}[fragile]{Важные моменты}
\begin{itemize}
    \item Использование неинициализированных указателей на функции и методы
        влечёт неопределённое поведение.

    \item В шаблонном коде указатель на функцию ведёт себя так же, как
        и объект класса с оператором \texttt{()}. Это позволяет использовать
        указатели на функции в качестве функторов.

    \item Для использования указателей на методы и поля классов нужны экземпляры
        этих классов.

    \item Указатели на методы и поля класса ни к чему не приводятся
        (используется для safe bool).

    \item Используйте \texttt{typedef}! $=)$.
\end{itemize}
\end{frame}

\end{document}



